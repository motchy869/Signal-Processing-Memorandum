\part{その他}
    \chapter{信号値の統計}
        信号に対し,その値の統計を定義できる場合がある。
        ここでは,連続時間の正弦波信号に対してその値の統計を考える。
        \section{連続時間正弦波信号値の確率密度関数}
            \begin{shadebox}
                周期 $T>0$, 振幅 $A>0$ の正弦波信号 $x:t\in\realNumbers\mapsto A\sin\parens{2\pi t/T}$ を考える。
                $x$ の確率密度関数 $p$ は次式である。
                \[
                    p(x) = \begin{cases}
                        \frac{1}{A\pi\sqrt{1-(x/A)^2}} & \parens{\abs{x} \leq A} \\
                        0 & \parens{\abs{x} > A} \\
                    \end{cases}
                \]
            \end{shadebox}
            \begin{proof}
                \quad\par
                1周期の範囲で統計を求めればよい。
                累積分布関数 $P(x) = \abs{\{t|-T/2\leq t\leq T/2, x(t) \leq x\}}/T$ を考える。
                まず $P(x) = 0\;(x < -A),\;P(x) = 1\;(x > A)$ は明らか。
                \par
                $-A \leq x \leq 0$ の場合を考える。
                $x(t) \leq u\;(-A \leq u \leq 0)$ を満たす $t\in[-T/2,T/2]$ は $-T/2 - \frac{T}{2\pi}\arcsin\parens{u/A}\leq t \leq \frac{T}{2\pi}\arcsin\parens{u/A}$ を満たす全ての $t$ であるから,$P(x) = 1/2 + \frac{1}{\pi}\arcsin\parens{x/A}\;(-A\leq x\leq 0)$ である。
                次に$0 \leq x \leq A$ の場合を考えると,$x$ のグラフの対称性から $P(x) = 1 - P(-x) = 1/2 + \frac{1}{\pi}\arcsin\parens{x/A}\;(0 \leq x \leq A)$ であることが解る。
                結局 $P(x) = 1/2 + \frac{1}{\pi}\arcsin\parens{x/A}\;(\abs{x}\leq A)$ である。
                これを $x$ について微分して $p$ を得る。
            \end{proof}
        \section{連続時間正弦波信号値の分散}
            \begin{shadebox}
                周期 $T>0$, 振幅 $A>0$ の正弦波信号 $x:t\in\realNumbers\mapsto A\sin\parens{2\pi t/T}$ を考える。
                $x$ の分散は $A^2/2$ である。
            \end{shadebox}
            \begin{proof}
                \quad\par
                求めたい分散は次式である。
                \[
                    \integrate{-A}{A}{x^2 p(x)}{}{x} = 2\integrate{0}{A}{x^2 p(x)}{}{x} = \integrate{0}{A}{\frac{2 x^2}{\pi A\sqrt{1-(x/A)^2}}}{}{x} \tag{1}
                \]
                ここで
                \[
                    \derivLong{\sqrt{1-(x/A)^2}}{x}{} = -\frac{x}{A^2\sqrt{1-(x/A)^2}}
                \]
                を用いて (1) を変形して次式を得る。
                \[ (1) = \integrate{0}{A}{\parens*{\derivLong{\sqrt{1-(x/A)^2}}{x}{}}\parens{-2x A/\pi}}{}{x} = \bracks*{\sqrt{1-(x/A)^2}\parens{-2x A/\pi}}_0^A + \integrate{0}{A}{\sqrt{1-(x/A)^2}\frac{2A}{\pi}}{}{x} \]
                $x = A\sin\theta$ なる変数変換を用いて $(1) = A^2/2$ を得る。
            \end{proof}
    \chapter{Heavisideの階段関数}
        \section{積分表示}
            \begin{shadebox}
                $H$をHeavisideの階段関数とする。
                次式が成り立つ。
                \[ H(x) = \lim_{\varepsilon\to +0}\frac{1}{2\pi i}\integrate{-\infty}{\infty}{\frac{1}{t-i\varepsilon}\NapierE^{ixt}}{}{t} \]
            \end{shadebox}
            \begin{proof}
                \quad\par
                複素積分を用いて示す。
                $R>\varepsilon,\;f(z) := \NapierE^{ixz}/(z-i\varepsilon)$とする。
                $f$の極は$i\varepsilon$であり,位数1, 留数1である。
                $x>0$のとき,積分路を$C_\mathrm{A}: C_1 + C_2,\; C_1 := [-R,R],\; C_2 := Re^{i\theta},\;\theta:0\to\pi$として$f$を$C_\mathrm{A}$上で積分する。
                留数定理から次式が成り立つ。
                \[ \integrate{C_\mathrm{A}}{}{f(z)}{}{z} = 2\pi i \quad \therefore \integrate{-R}{R}{f(z)}{}{z} = 2\pi i - \integrate{C_2}{}{f(z)}{}{z} \]
                \cite{数学備忘録}\ref{1:ベクトルLaplacianの発散}と同様にして$\integrate{C_2}{}{f(z)}{}{z} \to 0 \text{ as }R\to\infty$であるので$\lim_{R\to\infty}\integrate{-R}{R}{f(z)}{}{z} = 2\pi i$である。
                \par
                $x<0$のとき,積分路を$C_\mathrm{B}: -C_1 + C_3,\; C_3 := Re^{i\theta},\;\theta:-\pi\to 0$として$f$を$C_\mathrm{B}$上で積分する。
                $C_\mathrm{B}$が囲む領域に$f$の極が無いので,留数定理から次式が成り立つ。
                \[ \integrate{C_\mathrm{B}}{}{f(z)}{}{z} = 0 \quad \therefore \integrate{-R}{R}{f(z)}{}{z} = \integrate{C_3}{}{f(z)}{}{z} \]
                $C_\mathrm{2}$上の積分の評価と同様にして$\integrate{C_3}{}{f(z)}{}{z} \to 0 \text{ as }R\to\infty$であるので$\lim_{R\to\infty}\integrate{-R}{R}{f(z)}{}{z} = 0$である。
                以上より定理の主張が従う。
            \end{proof}
