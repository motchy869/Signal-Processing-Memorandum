\part{Laplace変換}
	\chapter{複素指数関数入力に対する伝達関数の作用}
		\begin{shadebox}
			$A>0,\;\omega \in \realNumbers$とする。
			連続時間信号$f: \realNumbers \to \complexNumbers$を次のように定める。
			\[
				f(t) =
				\begin{cases}
					A\NapierE^{i\omega t} & (t\geq 0) \\
					0 & (t<0)
				\end{cases}
			\]
			$H: s\in\complexNumbers \mapsto H(s) \in \complexNumbers$をproperで既約な有理関数とする。
			また,$H$の極の実部は全て負であるとする。
			伝達関数が$H(s)$である連続時間システムに信号$f$を入力した時の出力を$g$とすると,十分大きい$t$に対して
			$g(t) \sim H(i\omega)f(t)$となる。
		\end{shadebox}
		\begin{proof}
			\quad\par
			$N_\text{p}$を$H(s)$の分母多項式の相異なる零点の個数とし,それら零点を$p_0,\dots,p_{N_\text{p}}$とする。
			零点$p_k$の次数を$N_{\text{p},k}$とし,$H(s)$の部分分数展開を
			\[ H(s) = c_0 + \sum_{k=1}^{N_\mathrm{p}} \sum_{l=1}^{N_{\mathrm{p},k}} \frac{c_{k,l}}{(s-p_k)^l} \]
			とする。
			ここに$c_0,c_{k,l}\;(k=1,\dots,N_\mathrm{p},l=1,\dots,N_{\mathrm{p},k})$は適当な複素数である。
			$f,g$のLaplace変換をそれぞれ$F,G$とすると$G(s) = H(s)F(s) = A H(s)/(s-i\omega)$である。
			これの部分分数展開に現れる,$1/(s-p_k)^l\;(k=1,\dots,N_\mathrm{p},l=1,\dots,N_{\mathrm{p},k})$に比例する項は逆Laplace変換すると$t^{l-1}\NapierE^{p_k t}$に比例する関数となり,$t\to\infty$で0に収束する。\hfill \rule{10cm}{0.4pt}(1)
			\par
			残りの項,すなわち$1/(s-i\omega)$に比例する項は$AH(i\omega)/(s-i\omega) = H(i\omega)F(s)$となる。
			以上より,十分大きい$t$に対して$g(t) \sim \ILPLC{H(i\omega)F(s)}(t) = H(i\omega)f(t)$となる。
		\end{proof}
		\section{系: 正弦波入力に対する伝達関数の作用}
			\begin{shadebox}
				$A>0,\;\omega \in \realNumbers$とする。
				連続時間信号$f_1,f_2: \realNumbers \to \realNumbers$を次のように定める。
				\[
					f_1(t) =
					\begin{cases}
						A\cos\omega t & (t\geq 0) \\
						0 & (t<0)
					\end{cases}
				\]
				\[
					f_2(t) =
					\begin{cases}
						A\sin\omega t & (t\geq 0) \\
						0 & (t<0)
					\end{cases}
				\]
				$H$を直前の定理と同じように定める。
				伝達関数が$H(s)$である連続時間\textcolor{red}{実}システムに信号$f_1,f_2$を入力した時の出力をそれぞれ$g_1,g_2$とすると,十分大きい$t$に対して
				\begin{align*}
					g_1(t) &\sim |H(i\omega)|\cos(\omega t + \Arg\parens*{H(i\omega)}) \\
					g_2(t) &\sim |H(i\omega)|\sin(\omega t + \Arg\parens*{H(i\omega)})
				\end{align*}
				となる。
			\end{shadebox}
			\begin{proof}
				\quad\par
				$f_1$について示す。
				$f_2$も同様に示せる。
				$f_1(t) = \Re{A\NapierE^{i\omega t}}$であり,実数システムだから出力は$A\NapierE^{i\omega t}$を入力したときの出力の実部と等しい。
				直前の定理の結果を用いて
				\[ g_1(t) = \Re{H(i\omega)A\NapierE^{i\omega t}} = \Re{|H(i\omega)|\NapierE^{i\Arg\parens*{H(i\omega)}}A\NapierE^{i\omega t}} = |H(i\omega)|\cos (\omega t + \Arg\parens*{H(i\omega)}) \]
			\end{proof}
			\begin{proof}
				(直接的な証明)
				\quad\par
				$f_1$について示す。
				$f_2$も同様に示せる。
				直前の定理の証明の(1)までは同じである。
				$f_1,g_1$のLaplace変換をそれぞれ$F_1,G_1$とすると
				\[ F_1(s) = \frac{As}{s^2+\omega^2} = \frac{A}{2}\left(\frac{1}{s+i\omega} + \frac{1}{s-i\omega}\right) \]
				であるから,$G_1(s) = H(s)F(s)$の部分分数展開のうち$1/(s+i\omega),\;1/(s-i\omega)$に比例する項を詳しく調べれば良い。
				$1/(s+i\omega)$の係数は
				\[ \left. G(s)X(s)(s+i\omega) \right|_{s\to-i\omega} = AG(-i\omega)/2\]
				となり,$1/(s-i\omega)$の係数は
				\[ \left. G(s)X(s)(s-i\omega) \right|_{s\to i\omega} = AG(i\omega)/2\]
				となる。
				よってこれらの項の和は
				\begin{align}
					&\quad \frac{AG(-i\omega)/2}{s+i\omega} + \frac{AG(i\omega)/2}{s-i\omega} = \frac{A}{2}\left(\frac{G(-i\omega)}{s+i\omega} + \frac{G(i\omega)}{s-i\omega}\right) \nonumber\\
					&= \frac{A}{2}\times\frac{1}{s^2+\omega^2}\left(G(-i\omega)(s-i\omega) + G(i\omega)(s+i\omega)\right) \nonumber\\
					&= \frac{As}{s^2+\omega^2}\times\frac{1}{2}(G(i\omega)+G(-i\omega)) + \frac{A\omega}{s^2+\omega^2}\times\frac{-1}{2i}(G(i\omega)-G(-i\omega))
				\end{align}
				$G(s)$は有理式なので$G(-i\omega) = \conj{G(i\omega)}$となることに注意して
				\[ \frac{1}{2}(G(i\omega)+G(-i\omega)) = |G(i\omega)|\frac{1}{2}\left(\NapierE^{i\Arg\parens*{G(i\omega)}} + \NapierE^{-i\Arg\parens*{G(i\omega)}}\right) = |G(i\omega)|\cos\Arg\parens*{G(i\omega)} \]
				同様に
				\[ \frac{-1}{2i}(G(i\omega)-G(-i\omega)) = -|G(i\omega)|\sin\Arg\parens*{G(i\omega)} \]
				以上より,
				\begin{align*}
					(1) &= |G(i\omega)|\left(\cos\Arg\parens*{G(i\omega)}\frac{As}{s^2+\omega^2} - \sin\Arg\parens*{G(i\omega)}\frac{A\omega}{s^2+\omega^2} \right) \\
					g(t) &\sim \ILPLC{(1)}(t) = |G(i\omega)|\left(\cos\Arg\parens*{G(i\omega)}\cos\omega t - \sin\Arg\parens*{G(i\omega)}\sin\omega t \right) \\
					&= |G(i\omega)|\cos\left(\omega t + \Arg\parens*{G(i\omega)}\right)
				\end{align*}
			\end{proof}