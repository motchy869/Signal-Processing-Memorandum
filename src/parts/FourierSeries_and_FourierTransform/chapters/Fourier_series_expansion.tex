\chapter{Fourier級数展開}
    \section{基底関数}
        Fourier級数展開の基底関数はFourier変換やDFTのものと違って正規化されていないため,美しさに欠ける。
        \par
        $d\in\naturalNumbers,\;W_l>0\;(l=1,2,,\dots,d),\;\bm{k}\in\integers^d$とする。
        次式で定義される,$\bm{x}\in\realNumbers^d$に関する連続座標信号を,区間$\prod_{l=1}^d [-W_l,W_l]$に於ける第$\bm{k}$基底関数という。
        \[ W(\bm{k},\bm{x}) \coloneq \exp i\sum_{l=1}^d k_l\frac{x_l}{W_l}\pi \]

    \section{Fourier係数}
        $d\in\naturalNumbers,\;W_l>0\;(l=1,2,,\dots,d),\;\Omega\coloneq\prod_{l=1}^d [-W_l,W_l],\;\bm{k}\in\integers^d$とする。
        $f:\bm{x}\in\realNumbers \mapsto f(\bm{x})\in\realNumbers$を,第$l$座標に関して周期が$2W_l$であるような周期関数とする。
        次式で定義する,$\bm{k}$に関する離散座標信号を$f$の第$\bm{k}$ Fourier係数という。
        \[ c(f,\bm{k}) \coloneq \left(\prod_{l=1}^d 2W_l\right)^{-1}\integrate{\Omega}{}{\conj{W(\bm{k},\bm{x})}f(\bm{x})}{}{\bm{x}} \]