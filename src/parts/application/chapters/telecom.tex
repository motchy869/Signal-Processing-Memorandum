\chapter{通信への応用}
    \section{直交復調}
        \newcommand*{\fc}{f_\text{c}}
        \subsection{直交復調は正の周波数側にある信号を取り出して中心周波数を 0 にする}
            キャリア周波数を $\fc$ ,入力である実時間信号を $x:\realNumbers\to\realNumbers$ とすると,直交復調器は $x(t)\parens*{\cos(2\pi \fc t) - i\sin(2\pi \fc t)}$ を LPF に通してベースバンドの外側の周波数成分を取り除く。
            その結果,ベースバンド信号として $X = \mathcal{F}(x)$ の正の周波数側の信号(複素数値信号)が得られる。
            \par
            そうなる理由を説明する。
            LPF に入力される前の信号の周波数表示された Fourier 変換は次式である。
            \[ \parens*{X*(\delta_{-\fc} + \delta_{\fc})/2}(f) - i\parens*{X*(\delta_{-\fc} - \delta_{\fc})/(2i)}(f) \]
            ここに $\delta_a\;(a\in\realNumbers)$ は Dirac のデルタ関数を時間軸方向に $-a$ だけシフトしたものである。
            計算を進めると次式を得る。
            \[ \frac{1}{2}\parens*{X(f-\fc)+X(f+\fc)} - \frac{1}{2}\parens*{X(f-\fc)-X(f+\fc)} = X(f+\fc) \]
            $x$ は実数値関数であるから $X$ は Hermite 対称である。
            つまり適当な $X_+:\realNumbers\to\complexNumbers$ が存在して,$X(f) = X_+(f) + \conj{X_+(-f)}$ である。
            さらに $X_+$ の台は $\fc$ を中心とするベースバンド帯域幅に制限されている。
            よって $X(f+\fc)$ を LPF に通した結果は $X_+(f)$ である。
    \section{Nyquist ISI 基準}
        これは大雑把に言うと Fourier 変換が存在する連続時間信号 $h:\realNumbers\to\complexNumbers$ が時刻 0 を除いて,ある周期 $\Ts>0$ (s は symbol の意味)の整数倍の時刻で 0 になるための必要十分条件である。
        限定された周波数帯域を使って通信する際に受信側で情報を正しく復元するために重要な性質であり,詳細は \cite{Nyquist_ISI_crit} にある。
        数式で表すと次である。
        \[
            h(n\Ts) = \begin{cases}
                1 & n=0 \\
                0 & n\in\integers\setminus\{0\}
            \end{cases}
            \iff \forall f\in\realNumbers,\;\frac{1}{\Ts}\sum_{n=-\infty}^\infty H(f-n/\Ts) = 1
        \]
        ここに $H$ は $h$ の Fourier 変換である。
        \cite{Nyquist_ISI_crit} には $\Rightarrow$ の証明のみがある。
        本書では $\Leftarrow$ を証明する。
        \begin{proof}
            \begin{align*}
                1 &= \frac{1}{\Ts}\sum_{n=-\infty}^\infty H(f-n/\Ts) = \frac{1}{\Ts}\sum_{n=-\infty}^\infty\integrate{-\infty}{\infty}{h(t)\exp\parens*{-i2\pi(f-n/\Ts)t}}{}{t} \\
                &= \frac{1}{\Ts}\integrate{-\infty}{\infty}{h(t)\exp(-i2\pi ft)\sum_{n=-\infty}^\infty\exp\parens*{i2\pi nt/\Ts}}{}{t} \tag{1}
            \end{align*}
            ここで次の関係式を使う(\ref{定数関数1のDTFT} の派生版)。
            \[ \sum_{n=-\infty}^\infty\exp\parens*{i2\pi nt/\Ts} = 2\pi\Ts\sum_{n=-\infty}^\infty\delta(2\pi t-2\pi\Ts n) = \Ts\sum_{n=-\infty}^\infty\delta(t-n\Ts) \]
            これを式 (1) に適用して次式を得る。
            \begin{align*}
                1 &= \integrate{-\infty}{\infty}{h(t)\exp\parens*{-i2\pi ft}\sum_{n=-\infty}^\infty\delta(t-n\Ts)}{}{t} = \sum_{n=-\infty}^\infty\integrate{-\infty}{\infty}{h(t)\exp\parens*{-i2\pi ft}\delta(t-n\Ts)}{}{t} \\
                &= \sum_{n=-\infty}^\infty h(n\Ts)\exp\parens*{-i2\pi fn\Ts}
            \end{align*}
            右辺は $f$ に関する周期 $1/\Ts$ の関数の Fourier 級数であり,$h(n\Ts)$ は Fourier 係数である。
            左辺が 1 であることから $h(0) = 1,\;h(n\Ts)\;(n\neq 0) = 0$ である(より丁寧に論じるなら,前記の式の両辺に $\exp(i2\pi fk\Ts)\;(k\in\integers)$ を掛けて区間 $[-1/(2\Ts),1/(2\Ts)]$ で積分する。その結果が $k$ にどう依存するかを調べる)。
        \end{proof}
    \section{帯域制限された信号が一定時間間隔で無限に配置されると定数になる}
        \begin{shadebox}
            $T>0$ とする。
            連続時間信号 $h:\realNumbers\to\complexNumbers$ の Fourier 変換 $H$ の台が有界であり,$H(f)=0\;(\abs{f}\geq 1/T)$ であるとき,次が成り立つ。
            \[ \sum_{n=-\infty}^\infty h(t-nT) = H(0)/T \]
        \end{shadebox}
        例えば位相変調による通信の目的で設計された回路に於いて,シンボル周期と同じ時間間隔で同じ大きさのパルスを Raised-Cosine フィルタに入力し続けると出力は一定の値になる。
        直感的には Raised-Cosine フィルタのインパルス応答が見えるように思えるが,そうはならない。
        \begin{proof}
            \begin{align*}
                \sum_{n=-\infty}^\infty h(t-nT) &= \sum_{n=-\infty}^\infty \mathcal{F}^{-1}(H)(t-nT) = \sum_{n=-\infty}^\infty\integrate{-\infty}{\infty}{H(f)\exp(i2\pi f(t-nT))}{}{f} \\
                &= \integrate{-\infty}{\infty}{H(f)\exp(i2\pi ft)\sum_{n=-\infty}^\infty\exp(-i2\pi fnT)}{}{f} \tag{1}
            \end{align*}
            ここで次の関係式を使う(\ref{定数関数1のDTFT} の派生版)。
            \begin{align*}
                \sum_{n=-\infty}^\infty\exp(-i2\pi fnT) &= \frac{2\pi}{T}\sum_{n=-\infty}^\infty\delta(-2\pi f - 2\pi n/T) = \frac{1}{T}\sum_{n=-\infty}^\infty\delta(-f - n/T) \\
                &= \frac{1}{T}\sum_{n=-\infty}^\infty\delta(f + n/T) = \frac{1}{T}\sum_{n=-\infty}^\infty\delta(f - n/T)
            \end{align*}
            これを式 (1) に適用して次式を得る。
            \begin{align*}
                \sum_{n=-\infty}^\infty h(t-nT) &= \frac{1}{T}\sum_{n=-\infty}^\infty\integrate{-\infty}{\infty}{H(f)\exp(i2\pi ft)\delta(f-n/T)}{}{f} \\
                &= \frac{1}{T}\sum_{n=-\infty}^\infty H(n/T)\exp(i2\pi nt/T) = H(0)/T
            \end{align*}
            最後の等号は $h$ の帯域制限の前提による。
        \end{proof}