\part{その他}
    \chapter{Heavisideの階段関数}
        \section{積分表示}
            \begin{shadebox}
                $H$をHeavisideの階段関数とする。
                次式が成り立つ。
                \[ H(x) = \lim_{\varepsilon\to +0}\frac{1}{2\pi i}\integrate{-\infty}{\infty}{\frac{1}{t-i\varepsilon}e^{ixt}}{}{t} \]
            \end{shadebox}
            \begin{proof}
                \quad\par
                複素積分を用いて示す。
                $R>\varepsilon,\;f(z) := e^{ixz}/(z-i\varepsilon)$とする。
                $f$の極は$i\varepsilon$であり、位数1, 留数1である。
                $x>0$のとき、積分路を$C_\mathrm{A}: C_1 + C_2,\; C_1 := [-R,R],\; C_2 := Re^{i\theta},\;\theta:0\to\pi$として$f$を$C_\mathrm{A}$上で積分する。
                留数定理から次式が成り立つ。
                \[ \integrate{C_\mathrm{A}}{}{f(z)}{}{z} = 2\pi i \quad \therefore \integrate{-R}{R}{f(z)}{}{z} = 2\pi i - \integrate{C_2}{}{f(z)}{}{z} \]
                \cite{数学備忘録}\ref{1:ベクトルLaplacianの発散}と同様にして$\integrate{C_2}{}{f(z)}{}{z} \to 0 \text{ as }R\to\infty$であるので$\lim_{R\to\infty}\integrate{-R}{R}{f(z)}{}{z} = 2\pi i$である。
                \par
                $x<0$のとき、積分路を$C_\mathrm{B}: -C_1 + C_3,\; C_3 := Re^{i\theta},\;\theta:-\pi\to 0$として$f$を$C_\mathrm{B}$上で積分する。
                $C_\mathrm{B}$が囲む領域に$f$の極が無いので、留数定理から次式が成り立つ。
                \[ \integrate{C_\mathrm{B}}{}{f(z)}{}{z} = 0 \quad \therefore \integrate{-R}{R}{f(z)}{}{z} = \integrate{C_3}{}{f(z)}{}{z} \]
                $C_\mathrm{2}$上の積分の評価と同様にして$\integrate{C_3}{}{f(z)}{}{z} \to 0 \text{ as }R\to\infty$であるので$\lim_{R\to\infty}\integrate{-R}{R}{f(z)}{}{z} = 0$である。
                以上より定理の主張が従う。
            \end{proof}