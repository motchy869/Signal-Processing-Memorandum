 \part{畳み込み}
		\chapter{巡回畳み込み}
		$f,g$を、定義域が$\Omega := \{0,1,\cdots,N_1-1\}\times\{0,1,\cdots,N_2-1\}\times\cdots\times\{0,1,\cdots,N_d-1\}$であるような離散座標信号$f,g: \Omega\to\field;\;\bm{n} = [n_1,n_2,\cdots,n_d]^\top \mapsto f(\bm{n}),g(\bm{n})$とする。
		$\bm{N} := [N_1,\cdots,N_d]^\top$とする。
		$f$と$g$の巡回畳み込み$\cycConv{f}{g}$を次式で定義する。
		\[ \left(\cycConv{f}{g}\right)(\bm{n}) := \sum_{\bm{m} \in\Omega} f(\bm{m})g((\bm{n}-\bm{m})\%\bm{N}) \]

		\section{巡回畳み込みの可換則}
			\begin{shadebox}
				$f,g$を、定義域が$\Omega := \{0,1,\cdots,N_1-1\}\times\{0,1,\cdots,N_2-1\}\times\cdots\times\{0,1,\cdots,N_d-1\}$であるような離散座標信号$f,g: \Omega\to\field;\;\bm{n} = [n_1,n_2,\cdots,n_d]^\top \mapsto f(\bm{n}),g(\bm{n})$とするとき、次が成り立つ。
				\[ \cycConv{f}{g} = \cycConv{g}{f} \]
			\end{shadebox}
			\begin{proof}
				\begin{align}
					\left(\cycConv{g}{f}\right)(\bm{n}) &= \sum_{\bm{m}\in\Omega} g(\bm{m})f((\bm{n}-\bm{m})\%\bm{N}) \nonumber \\
					&= \sum_{m_1=0}^{N_1-1}\sum_{\bm{m}_2\in\Omega_2}g(m_1,\bm{m}_2)f((n_1 - m_1)\%N_1,(\bm{n_2}-\bm{m_2})\%\bm{N_2})
				\end{align}
				ここに$\bm{n}_i := [n_i,\cdots,n_d]^\top\;(\bm{m}_i,\bm{N}_i\text{も同様}),\;\Omega_i := \{0,1,\cdots,N_i-1\}\times\cdots\times\{0,1,\cdots,N_d-1\}$である。
				\begin{align*}
					(2) &= \sum_{m_1=0}^{n_1}\sum_{\bm{m}_2\in\Omega_2}g(m_1,\bm{m}_2)f(n_1 - m_1,(\bm{n_2}-\bm{m_2})\%\bm{N_2}) \\
					&\quad + \sum_{m_1=n_1+1}^{N_1-1}\sum_{\bm{m}_2\in\Omega_2}g(m_1,\bm{m}_2)f(n_1 + N_1 - m_1,(\bm{n_2}-\bm{m_2})\%\bm{N_2}) \\
					&= \sum_{l_1=n_1}^0 \sum_{\bm{m}_2\in\Omega_2}g(n_1 - l_1,\bm{m}_2)f(l_1,(\bm{n_2}-\bm{m_2})\%\bm{N_2}) \\
					&\quad + \sum_{l_1=N_1-1}^{n_1+1}\sum_{\bm{m}_2\in\Omega_2}g(n_1+N_1-l_1,\bm{m}_2)f(l_1,(\bm{n_2}-\bm{m_2})\%\bm{N_2}) \\
					&= \sum_{l_1=n_1}^0 \sum_{\bm{m}_2\in\Omega_2}g(\textcolor{darkpastelgreen}{(n_1-l_1)\%N_1},\bm{m}_2)f(l_1,(\bm{n_2}-\bm{m_2})\%\bm{N_2}) \\
					&\quad + \sum_{l_1=N_1-1}^{n_1+1}\sum_{\bm{m}_2\in\Omega_2}g((n_1-l_1)\%N_1,\bm{m}_2)f(l_1,(\bm{n_2}-\bm{m_2})\%\bm{N_2}) \\
					&= \sum_{l_1=0}^{N_1-1} \sum_{\bm{m}_2\in\Omega_2}g((n_1-l_1)\%N_1,\bm{m}_2)f(l_1,(\bm{n_2}-\bm{m_2})\%\bm{N_2}) \\
				\end{align*}
				同様の変形を繰り返すと最終的に次のようになる。
				\[ \left(\cycConv{g}{f}\right)(\bm{n}) = \sum_{\bm{l}\in\Omega} g((\bm{n}-\bm{l})\%\bm{N})f(\bm{l}) = \left(\cycConv{f}{g}\right)(\bm{n}) \]
			\end{proof}
		\chapter{諸定理}
			\section{線形変換と畳み込みの順序交換}
				\subsection{動機}
					画像処理に於いてカーネルとの畳み込みを実行してから線形変換を施す場合と、事前に画像とカーネルの両方に線形変換を施してから畳み込む場合の結果の違いに関心がある。
				\subsection{理論}
					$d\in\naturalNumbers$とし、$f:\bm{x}\in\realNumbers^d\mapsto f(\bm{x})\in\realNumbers$を$d$次元信号とする。
					線形変換を表す正則行列を$A$とし、$A$による変換を$T_A$と表す。
					$T_A$による変換は次式を以て定義する。
					\[ T_A(f)(\bm{x}) = f(A^{-1}\bm{x}) \]
					$G:\bm{x}\in\realNumbers^d\mapsto G(\bm{x})\in\realNumbers$を$d$次元信号とする。
					このとき次式が成り立つ。
					\[ T_A(G)*T_A(f) = |A|T_A(G*f) \]
					\begin{proof}
						\quad\par
						$\mu$をJordan測度とする。
						\begin{align*}
							T_A(G)*T_A(f)(\bm{x}) &= \LebInteg{\realNumbers^d}{T_A(G)(\bm{x}-\bm{u})T_A(f)(\bm{u})}{\mu}{\bm{u}} = \LebInteg{\realNumbers^d}{G(A^{-1}(\bm{x}-\bm{u}))f(A^{-1}\bm{u})}{\mu}{\bm{u}} \\
							&= \LebInteg{\realNumbers^d}{G(A^{-1}\bm{x} - A^{-1}\bm{u})f(A^{-1}\bm{u})}{\mu}{\bm{u}} \\
							&= \LebInteg{\realNumbers^d}{G(A^{-1}\bm{x} - \bm{v})f(\bm{v})}{\abs{|A|}\mu}{\bm{v}} \\
							&\phantom{=} (\bm{v}=A^{-1}\bm{u}\text{と変数変換した。}\abs{|A|}\text{は}|A|の絶対値である。) \\
							&= \abs{|A|} \LebInteg{\realNumbers^d}{G(A^{-1}\bm{x} - \bm{v})f(\bm{v})}{\mu}{\bm{v}} \\
							&= \abs{|A|}T_A(G*f)(\bm{x})
						\end{align*}
					\end{proof}
				\section{数値実験}
					Mathematicaによる例が「線形変換と畳み込み.nb」にある。