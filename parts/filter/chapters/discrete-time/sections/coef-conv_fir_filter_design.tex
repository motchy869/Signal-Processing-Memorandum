\section{係数畳み込み型 FIR フィルタの設計}
    \subsection{DTFT の誤差 2 乗積分最小化}
        \newcommand{\Hideal}{H_\text{ideal}}
        \newcommand{\vhOpt}{\bm{h}_\text{opt}}
        ここではフィルタ係数が複素数であり、個数が指定された条件下で、所望の周波数特性に対して DTFT の誤差の 2 乗積分が最小になるように係数を決定する方法を記す。
        \par
        この方法は係数の個数について偶奇を区別せず、係数の添え字に関する対称性も指定していないため、線形位相特性は保証されない(意図的に線形位相特性を狙う方法については \cite{learn_sp_from_basic} 「8.3.1 線形位相 FIR フィルタの設計法」に詳細がある)。
        線形位相特性に関心が無く、とにかく所望の周波数特性に近付けることを第一とする場面を想定している。
        \par
        この方法で作られたフィルタの振幅周波数特性にはリップルが生じる可能性がある(激しさは所望の周波数特性の不連続性の強さに依る)。
        その際の緩和策として \cite{learn_sp_from_basic} 「8.3.2 窓関数を用いた設計法」が適用できると期待される。
        \begin{shadebox}
            $N\in\naturalNumbers$ 個の複素数 $h_0,h_1,\dots,h_{N-1}$ を係数とする FIR フィルタの周波数特性(フィルタ係数列の DTFT)を $H:\realNumbers\to\complexNumbers$ と記し、所望の周波数特性を $\Hideal:\realNumbers\to\complexNumbers$ と記す。
            但し $H$ と $\Hideal$ の引数はサンプル周波数について正規化された角周波数である($\pi/2$ が Nyquist 周波数に対応する)。
            $\bm{h} \coloneq [h_0,h_1,\dots,h_{N-1}]^\top$ として次式で定義される $H$ と $\Hideal$ の誤差 2 乗積分の評価関数
            \[ J(\bm{h}) \coloneq \frac{1}{2\pi}\integrate{-\pi}{\pi}{\abs{H(\Omega)-\Hideal(\Omega)}^2}{}{\Omega} \]
            の最小点 $\vhOpt$ は次式である。
            \[ \vhOpt[k] = \frac{1}{2\pi}\integrate{-\pi}{\pi}{\Hideal(\Omega)\NapierE^{i\Omega k}}{}{\Omega}\;(k=0,1,\dots,N-1) = \IDTFT{\Hideal}(k) \]
            とくに $\Hideal$ が Hermite 対称であるとき(実数値信号の周波数特性)は $\vhOpt\in\realNumbers^N$ となる。
        \end{shadebox}
        \begin{proof}
            \quad\par\noindent
            (最小点の導出)
            \begin{align*}
                J(\bm{h}) &= \frac{1}{2\pi}\integrate{-\pi}{\pi}{\parens*{H(\Omega)-\Hideal(\Omega)}\parens*{\conj{H(\Omega)-\Hideal(\Omega)}}}{}{\Omega} \\
                &= \frac{1}{2\pi}\integrate{-\pi}{\pi}{\parens*{\sum_{k=0}^{N-1} h_k\NapierE^{-i\Omega k} -\Hideal(\Omega)}\parens*{\sum_{l=0}^{N-1} \conj{h_l}\NapierE^{i\Omega l} -\conj{\Hideal(\Omega)}}}{}{\Omega} \\
                &= \frac{1}{2\pi}\sum_{k=0}^{N-1}\sum_{l=0}^{N-1} h_k\conj{h_l}\integrate{-\pi}{\pi}{\NapierE^{i\Omega(l-k)}}{}{\Omega} - \frac{1}{2\pi}\sum_{k=0}^{N-1} h_k\integrate{-\pi}{\pi}{\conj{\Hideal(\Omega)}\NapierE^{-i\Omega k}}{}{\Omega} \\
                &\phantom{=}- \frac{1}{2\pi}\sum_{l=0}^{N-1} \conj{h_l}\integrate{-\pi}{\pi}{\Hideal(\Omega)\NapierE^{i\Omega l}}{}{\Omega} + \frac{1}{2\pi}\integrate{-\pi}{\pi}{\abs{\Hideal(\Omega)}^2}{}{\Omega} \\
                &= \norm{\bm{h}}_2^2 - \frac{1}{2\pi}2\Re{\sum_{k=0}^{N-1} h_k\integrate{-\pi}{\pi}{\conj{\Hideal(\Omega)}\NapierE^{-i\Omega k}}{}{\Omega}} + \frac{1}{2\pi}\integrate{-\pi}{\pi}{\abs{\Hideal(\Omega)}^2}{}{\Omega} \tag{1}
            \end{align*}
            ここで $v_k\;(k=0,1,\dots,N-1)$ を次式で定義し、$\bm{v}\coloneq[v_0,v_1,\dots,v_{N-1}]^\top$ とする。
            \[ v_k \coloneq \integrate{-\pi}{\pi}{\Hideal(\Omega)\NapierE^{i\Omega k}}{}{\Omega} \]
            式 (1) の第 3 項は $\bm{h}$ に依存しないので無視し、第 2 項までを取り出して次の評価関数を得る。
            \[ J_2(\bm{h}) \coloneq \norm{\bm{h}}_2^2 - \frac{1}{2\pi}2\Re{\bm{v}^*\bm{h}} = \norm{\bm{h} - \frac{1}{2\pi}\bm{v}}_2^2 - \frac{1}{4\pi^2}\norm{\bm{v}}_2^2 \]
            これの最小点は $\bm{v}/(2\pi)$ であるから主張が従う。
            \newline
            \par\noindent
            ($\Hideal$ が Hermite 対称であるときは $\vhOpt\in\realNumbers^N$ となること)
            \[ \vhOpt[k] = \frac{1}{2\pi}\bracks*{\underbrace{\integrate{-\pi}{0}{\Hideal(\Omega)\NapierE^{i\Omega k}}{}{\Omega}}_{(2)} + \integrate{0}{\pi}{\Hideal(\Omega)\NapierE^{i\Omega k}}{}{\Omega}} \]
            式 (2) に変数変換を施すと次式を得る。
            \[ (2) = \integrate{0}{\pi}{\Hideal(-\Omega)\NapierE^{-i\Omega k}}{}{\Omega} = \conj{\integrate{0}{\pi}{\Hideal(\Omega)\NapierE^{i\Omega k}}{}{\Omega}} \]
            よって次式を得る。
            \[ \vhOpt[k] = \frac{1}{\pi}\Re{\integrate{0}{\pi}{\Hideal(\Omega)\NapierE^{i\Omega k}}{}{\Omega}}\in\realNumbers \]
        \end{proof}
