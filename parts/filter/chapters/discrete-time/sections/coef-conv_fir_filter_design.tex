\section{係数畳み込み型 FIR フィルタの設計}
    \subsection{DTFT の誤差 2 乗積分最小化}
        \newcommand{\Hideal}{H_\text{ideal}}
        \newcommand{\vhOpt}{\bm{h}_\text{opt}}
        ここではフィルタ係数が複素数であり,個数が指定された条件下で,所望の周波数スペクトラムに対して DTFT の誤差の重み付き 2 乗積分が最小になるように係数を決定する方法を記す。
        \par
        この方法は係数の個数について偶奇を区別せず,係数の添え字に関する対称性も指定していないため,線形位相特性は保証されない(意図的に線形位相特性を狙う方法については \cite{learn_sp_from_basic} 「8.3.1 線形位相 FIR フィルタの設計法」に詳細がある)。
        周波数スペクトラムの補正のための逆特性フィルタの設計のように,所望のフィルタの周波数スペクトラムが必ずしも線形位相特性を持たない場面を想定している。
        \par
        この方法で作られたフィルタの振幅周波数スペクトラムにはリップルが生じる可能性がある(激しさは所望の周波数スペクトラムの不連続性の強さに依る)。
        その際の緩和策として \cite{learn_sp_from_basic} 「8.3.2 窓関数を用いた設計法」が適用できると期待される。
        \subsubsection{誤差の重みが一様である場合}
            \label{誤差の重みが一様である場合}
            \newcommand{\Nm}{{N_\text{m}}}
            \newcommand{\Np}{{N_\text{p}}}
            ここでは全ての周波数に於いて誤差の重みを 1 としたときの最適係数を導く。
            \begin{shadebox}
                $\Nm,\Np\in\naturalNumbers\cup\{0\}$ とする。
                $N\coloneq\Nm+\Np+1$ 個の複素数 $h_{-\Nm},h_{-\Nm+1},\dots,h_{\Np-1},h_\Np$ を係数とする FIR フィルタの周波数スペクトラム(フィルタ係数列の DTFT)を $H:\realNumbers\to\complexNumbers$ と記し,所望の周波数スペクトラムを $\Hideal:\realNumbers\to\complexNumbers$ と記す。
                但し $H$ と $\Hideal$ の引数はサンプル周波数について正規化された角周波数である($\pi/2$ が Nyquist 周波数に対応する)。
                $\bm{h} \coloneq [h_{-\Nm},h_{-\Nm+1},\dots,h_{\Np-1},h_\Np]^\top$ として次式で定義される $H$ と $\Hideal$ の誤差 2 乗積分の評価関数
                \[ J(\bm{h}) \coloneq \frac{1}{2\pi}\integrate{-\pi}{\pi}{\abs{H(\Omega)-\Hideal(\Omega)}^2}{}{\Omega} \tag{1} \]
                の最小点 $\vhOpt$ の第 $n\in\naturalNumbers\cup\{0\}$ 要素は次式である。
                \[ \vhOpt[n] = \frac{1}{2\pi}\integrate{-\pi}{\pi}{\Hideal(\Omega)\NapierE^{i\Omega (n-\Nm)}}{}{\Omega} = \IDTFT{\Hideal}(n-\Nm) \]
                とくに $\Hideal$ が Hermite 対称であるとき(実数値信号の周波数スペクトラム)は $\vhOpt\in\realNumbers^N$ となる。
            \end{shadebox}
            \begin{proof}
                \quad\par\noindent
                (最小点の導出)
                \begin{align*}
                    J(\bm{h}) &= \frac{1}{2\pi}\integrate{-\pi}{\pi}{\parens*{H(\Omega)-\Hideal(\Omega)}\parens*{\conj{H(\Omega)-\Hideal(\Omega)}}}{}{\Omega} \\
                    &= \frac{1}{2\pi}\integrate{-\pi}{\pi}{\parens*{\sum_{k=-\Nm}^\Np h_k\NapierE^{-i\Omega k} -\Hideal(\Omega)}\parens*{\sum_{l=-\Nm}^\Np \conj{h_l}\NapierE^{i\Omega l} -\conj{\Hideal(\Omega)}}}{}{\Omega} \\
                    &= \frac{1}{2\pi}\sum_{k=-\Nm}^\Np\sum_{l=-\Nm}^\Np h_k\conj{h_l}\integrate{-\pi}{\pi}{\NapierE^{i\Omega(l-k)}}{}{\Omega} - \frac{1}{2\pi}\sum_{k=-\Nm}^\Np h_k\integrate{-\pi}{\pi}{\conj{\Hideal(\Omega)}\NapierE^{-i\Omega k}}{}{\Omega} \\
                    &\phantom{=}- \frac{1}{2\pi}\sum_{l=-\Nm}^\Np \conj{h_l}\integrate{-\pi}{\pi}{\Hideal(\Omega)\NapierE^{i\Omega l}}{}{\Omega} + \frac{1}{2\pi}\integrate{-\pi}{\pi}{\abs{\Hideal(\Omega)}^2}{}{\Omega} \\
                    &= \norm{\bm{h}}_2^2 - \frac{1}{2\pi}2\Re{\sum_{k=-\Nm}^\Np h_k\integrate{-\pi}{\pi}{\conj{\Hideal(\Omega)}\NapierE^{-i\Omega k}}{}{\Omega}} + \frac{1}{2\pi}\integrate{-\pi}{\pi}{\abs{\Hideal(\Omega)}^2}{}{\Omega} \tag{2}
                \end{align*}
                ここで $v_k\;(k=-\Nm,\;-\Nm+1,\;\dots,\;\Np)$ を次式で定義し,$\bm{v}\coloneq[v_{-\Nm},\;v_{-\Nm+1},\;\dots,\;v_\Np]^\top$ とする。
                \[ v_n \coloneq \frac{1}{2\pi}\integrate{-\pi}{\pi}{\Hideal(\Omega)\NapierE^{i\Omega n}}{}{\Omega} \]
                式 (2) の第 3 項は $\bm{h}$ に依存しないので無視し,第 2 項までを取り出して次の評価関数を得る。
                \[ J_2(\bm{h}) \coloneq \norm{\bm{h}}_2^2 - 2\Re{\bm{v}^*\bm{h}} = \norm{\bm{h} - \bm{v}}_2^2 - \norm{\bm{v}}_2^2 \]
                これの最小点が $\bm{v}$ であるから主張が従う。
                \newline
                \par\noindent
                ($\Hideal$ が Hermite 対称であるときは $\vhOpt\in\realNumbers^N$ となること)
                \[ \vhOpt[n] = \frac{1}{2\pi}\bracks*{\underbrace{\integrate{-\pi}{0}{\Hideal(\Omega)\NapierE^{i\Omega (n-\Nm)}}{}{\Omega}}_{(3)} + \integrate{0}{\pi}{\Hideal(\Omega)\NapierE^{i\Omega (n-\Nm)}}{}{\Omega}} \]
                式 (3) に変数変換を施すと次式を得る。
                \[ (3) = \integrate{0}{\pi}{\Hideal(-\Omega)\NapierE^{-i\Omega (n-\Nm)}}{}{\Omega} = \conj{\integrate{0}{\pi}{\Hideal(\Omega)\NapierE^{i\Omega (n-\Nm)}}{}{\Omega}} \]
                よって次式を得る。
                \[ \vhOpt[n] = \frac{1}{\pi}\Re{\integrate{0}{\pi}{\Hideal(\Omega)\NapierE^{i\Omega (n-\Nm)}}{}{\Omega}}\in\realNumbers \]
            \end{proof}
            \paragraph{因果的フィルタとして実現する方法}
                $\Nm>0$ のときは前記のフィルタ係数の添え字に負数が含まれ,因果的フィルタとして実現できない。
                最小の添え字が 0 になるように時間的にシフトする必要がある。
                フィルタの応答が時間的に遅れる(周波数領域に於いては遅れに比例した角周波数の複素指数関数が乗じられる)が,無線通信等の大抵の用途では問題にならない。
            \paragraph{Note:IDFT による係数設計との違い}
                \newcommand{\kmin}{{k_\text{min}}}
                \newcommand{\kmax}{{k_\text{max}}}
                \newcommand{\HidealNA}{H_\text{ideal,NA}}
                実用の場面では $\Hideal$ が等間隔の離散的な正規化角周波数についてのみ得られることがある。
                その場合に $\bm{h}$ の決定によく用いられているのは IDFT である。
                IDFT による結果と \ref{誤差の重みが一様である場合} の結果が一般には一致しないことを示しておく。
                \par
                $\Delta f>0,\;\kmin,\;\kmax\in\integers,\;\kmax>\kmin$ とする。
                $\Hideal$ の値が $k\Delta f\;k=\kmin,\kmin+1,\dots,\kmax$ に関して得られているとする。
                $\Hideal$ の Fourier 変換の台が $[-1/(2\Delta f),1/(2\Delta f)]$ に含まれいるものとする。
                標本化定理から次式が成り立つ。
                \[ \Hideal(f) = \sum_{k=\kmin}^\kmax \Hideal(k\Delta f)\sinc\frac{\pi}{\Delta f}(f-k\Delta f) \]
                サンプル周波数を $\fsamp$ として $\Hideal$ を正規化角周波数で表したものを $\HidealNA$ とすると次式を得る。
                \[ \HidealNA(\Omega) = \Hideal(\Omega\fsamp/(2\pi)) \]
                これに対して \ref{誤差の重みが一様である場合} の方法を適用すると次式が成り立つ。
                \begin{align*}
                    \vhOpt[n] &= \IDTFT{\HidealNA}(n-\Nm) = \sum_{k=\kmin}^\kmax \Hideal(k\Delta f)\IDTFT{\Omega\mapsto\sinc\frac{\pi}{\Delta f}\parens*{\frac{\Omega\fsamp}{2\pi}-k\Delta f}}(N-\Nm) \\
                    &= \sum_{k=\kmin}^\kmax \Hideal(k\Delta f)\frac{1}{2\pi}\integrate{-\pi}{\pi}{\bracks*{\sinc\frac{\pi}{\Delta f}\parens*{\frac{\Omega\fsamp}{2\pi}-k\Delta f}}(\exp i\Omega n)}{}{\Omega}
                \end{align*}
                これは一般には IDFT による結果と一致しない。
        \subsubsection{誤差の重みが一様ではない場合}
            ここでは,誤差の重みが周波数に依存する場合の最適係数を導く。
            \begin{shadebox}
                $w:\realNumbers\to\realNumbers$ は非負であり,$w>0$ となる区間が $[-\pi,\pi]$ に少なくとも 1 つ存在するものとする。
                式 (1) を変更し,次の評価関数を考える。
                \[ J(\bm{h}) \coloneq \frac{1}{2\pi}\integrate{-\pi}{\pi}{w(\Omega)\abs{H(\Omega)-\Hideal(\Omega)}^2}{}{\Omega}\tag{4} \]
                これの最小点 $\vhOpt$ は次式である。
                \[ \vhOpt = A^{-1}\bm{v} \]
                ここに $A\in\complexNumbers^{N\times N},\;\bm{v}\in\complexNumbers^N$ はそれぞれ次式で定義される。
                \[ A[m,n] \coloneq \frac{1}{2\pi}\integrate{-\pi}{\pi}{w(\Omega)\NapierE^{i\Omega(m-n)}}{}{\Omega},\quad v[m] \coloneq \frac{1}{2\pi}\integrate{-\pi}{\pi}{w(\Omega)\Hideal(\Omega)\NapierE^{i\Omega (m-\Nm)}}{}{\Omega} \]
                とくに $\Hideal$ が Hermite 対称であり(実数値信号の周波数スペクトラム),かつ $w$ が偶関数であるときは $\vhOpt\in\realNumbers^N$ となる。
            \end{shadebox}
            \begin{proof}
                \quad\par\noindent
                (最小点の導出)
                \par
                式 (1) と同じ要領で式変形を行うと次式を得る。
                \begin{align*}
                    J(\bm{h}) &= \frac{1}{2\pi}\integrate{-\pi}{\pi}{w(\Omega)\parens{H(\Omega)-\Hideal(\Omega)}\parens{\conj{H(\Omega)-\Hideal(\Omega)}}}{}{\Omega} \\
                    &= \frac{1}{2\pi}\sum_{k=-\Nm}^\Np\sum_{l=-\Nm}^\Np h_k\conj{h_l}\integrate{-\pi}{\pi}{w(\Omega)\NapierE^{i\Omega(l-k)}}{}{\Omega} - \frac{1}{2\pi}2\Re{\sum_{k=-\Nm}^\Np h_k\integrate{-\pi}{\pi}{w(\Omega)\conj{\Hideal(\Omega)}\NapierE^{-i\Omega k}}{}{\Omega}} \\
                    &\phantom{=} + \frac{1}{2\pi}\integrate{-\pi}{\pi}{w(\Omega)\abs{\Hideal(\Omega)}^2}{}{\Omega}\tag{5}
                \end{align*}
                $N\times N$ Hermite 対称行列 $A$ を,第 $(l,k)$ 要素が次式で定義されるものとする。
                \[ A[l,k] \coloneq \frac{1}{2\pi}\integrate{-\pi}{\pi}{w(\Omega)\NapierE^{i\Omega((l-\Nm)-(k-\Nm))}}{}{\Omega} = \frac{1}{2\pi}\integrate{-\pi}{\pi}{w(\Omega)\NapierE^{i\Omega(l-k)}}{}{\Omega} \]
                また $v_n\;(n=-\Nm,\;-\Nm+1,\;\dots,\;\Np)$ を次式で定義し,$\bm{v}\coloneq[v_{-\Nm},\;v_{-\Nm+1},\;\dots,\;v_\Np]^\top$ とする。
                \[ v_n \coloneq \frac{1}{2\pi}\integrate{-\pi}{\pi}{w(\Omega)\Hideal(\Omega)\NapierE^{i\Omega n}}{}{\Omega} \]
                上記を式 (5) に適用し,$\bm{h}$ に依存しない項を無視すると次の評価関数を得る。
                \[ J_2(\bm{h}) \coloneq \bm{h}^*A\bm{h} - 2\Re{\bm{v}^*\bm{h}} \]
                証明していないが $A$ は正定値であると期待され\footnote{$w$ が偶関数であれば,区間 $[-\pi,\pi]$ に於ける $w$ の Fourier 級数展開を考えると $A$ が巡回行列となって固有値が容易に計算でき,正定性を示せるように思える。試してはいない。},以下ではそう仮定する。
                すると $J_2(\bm{h})$ は $\bm{h}$ の狭義凸 2 次形式であり,最小点が存在する。
                それは $\bm{h}$ が微小量 $\Delta\bm{h}$ 変化するときの $J_2$ の増分 $\Delta J_2(\bm{h})$ が 0 となる $\bm{h}$ である。
                \[ \Delta J_2(\bm{h}) = \bm{h}^*A\Delta\bm{h} + \Delta\bm{h}^*A\bm{h} - 2\Re{\bm{v}^*\Delta\bm{h}} + \underset{\norm{\Delta\bm{h}}_2\to 0}{o}(\norm{\Delta\bm{h}}_2) = 2\Re{(\bm{h}^*A-\bm{v}^*)\Delta\bm{h}} \]
                $\Delta J_2(\bm{h}) = 0\iff\bm{h} = A^{-1}\bm{v}$ であるから主張が従う。
                \newline
                \par\noindent
                ($\Hideal$ が Hermite 対称であり(実数値信号の周波数スペクトラム),かつ $w$ が偶関数であるときは $\vhOpt\in\realNumbers^N$ となること)
                \par
                重みが一様である場合 (\cref{誤差の重みが一様である場合}) の導出と同じ要領で $A\in\realNumbers^{N\times N},\;\bm{v}\in\realNumbers^N$ であることが容易に示せて $\vhOpt=A^{-1}\bm{v}\in\realNumbers^N$ である。
            \end{proof}
    \subsection{DTFT の誤差の実部と虚部の∞ノルム最小化}
        \newcommand{\Ntp}{N_{\text{tp}}}
        \newcommand{\Hid}{H_{\text{id}}}
        \newcommand{\HidR}{H_{\text{id,r}}}
        \newcommand{\HidI}{H_{\text{id,i}}}
        \newcommand{\HidH}{H_{\text{id,H}}}
        \newcommand{\HidHr}{H_{\text{id,H,r}}}
        \newcommand{\HidHi}{H_{\text{id,H,i}}}
        \newcommand{\HidSH}{H_{\text{id,SH}}}
        \newcommand{\HidSHr}{H_{\text{id,SH,r}}}
        \newcommand{\HidSHi}{H_{\text{id,SH,i}}}
        \newcommand{\hes}{h_{\text{es}}}
        \newcommand{\hos}{h_{\text{os}}}
        \newcommand{\Hes}{H_{\text{es}}}
        \newcommand{\Hos}{H_{\text{os}}}
        \newcommand{\hr}{h_{\text{r}}}
        \newcommand{\hi}{h_{\text{i}}}
        \newcommand{\hrEs}{h_{\text{r,es}}}
        \newcommand{\hrOs}{h_{\text{r,os}}}
        \newcommand{\hiEs}{h_{\text{i,es}}}
        \newcommand{\hiOs}{h_{\text{i,os}}}
        \newcommand{\hesOpt}{h_{\text{es,opt}}}
        \newcommand{\hosOpt}{h_{\text{os,opt}}}
        \newcommand{\hopt}{h_{\text{opt}}}
        \newcommand{\hrEsOpt}{h_{\text{r,es,opt}}}
        \newcommand{\hrOsOpt}{h_{\text{r,os,opt}}}
        \newcommand{\hiEsOpt}{h_{\text{i,es,opt}}}
        \newcommand{\hiOsOpt}{h_{\text{i,os,opt}}}
        \newcommand{\HesOpt}{H_{\text{es,opt}}}
        \newcommand{\HosOpt}{H_{\text{os,opt}}}
        \newcommand{\Hopt}{H_{\text{opt}}}
        \newcommand{\HrEsOpt}{H_{\text{r,es,opt}}}
        \newcommand{\HrOsOpt}{H_{\text{r,os,opt}}}
        \newcommand{\HiEsOpt}{H_{\text{i,es,opt}}}
        \newcommand{\HiOsOpt}{H_{\text{i,os,opt}}}
        \newcommand{\eEs}{\mathrm{e}_{\text{es}}}
        \newcommand{\eOs}{\mathrm{e}_{\text{os}}}
        \newcommand{\eOpt}{\mathrm{e}_{\text{opt}}}
        \newcommand{\erEs}{\mathrm{e}_{\text{r,es}}}
        \newcommand{\erOs}{\mathrm{e}_{\text{r,os}}}
        \newcommand{\eiEs}{\mathrm{e}_{\text{i,es}}}
        \newcommand{\eiOs}{\mathrm{e}_{\text{i,os}}}
        フィルタ係数が複素数であり,個数が指定された条件下で,所望の周波数スペクトラムに対して DTFT の誤差の実、虚部それぞれの偶、奇対称成分の $L^\infty$-ノルムが最小になるように係数を決定する方法を記す。
        $H$ が線形位相特性をもち,かつ偶対称か奇対称のいずれかである場合(★ 1)には Remez のアルゴリズムが知られている(例えば \cite{DSP_JL} の \inlineCode{remez} 関数)。
        ここでは $H$ が★ 1 の条件を満たさない場合にも Remez のアルゴリズムを 4 回独立に実行してフィルタ係数を決定する方法を示す。
        また特殊ケースとして,入力信号が実数(従ってフィルタ係数も実数)である場合に Remez のアルゴリズムを 2 回独立に実行してフィルタ係数を決定する方法も示す。
        議論の見通しを良くするため,まず特殊ケースを導き,これを拡張して最終的な結論を導く。
        \subsubsection{前提}
            以下では周波数スペクトラムの補償用フィルタのように, null を含まない穏やかな特性を持つフィルタを主な関心の対象としている。
            そのため以下ではフィルタ係数の個数 $\Ntp\in\naturalNumbers$ は奇数であるとする。
            そうでない場合も以下で述べる手法がそのまま使える。
        \subsubsection{信号が実数の場合}
            まず信号が実数であり,従ってフィルタ係数も実数である場合について述べる。
            入力信号が実数であるから $\Hid$ は Hermite 対称である。
            \paragraph{偶、奇対称の係数がもたらす周波数スペクトラム}
                \label{偶、奇対称の係数がもたらす周波数スペクトラム}
                Remez のアルゴリズムの適用を考える準備として,添え字に関して偶対称と奇対称な係数がもたらす周波数スペクトラムを見ておく。
                いま,$\hes:\integers\to\realNumbers$ と $\hos:\integers\to\realNumbers$ をそれぞれ次式を満たす偶対称と奇対称(「対称」とは,中央の係数を中心にして言う)のフィルタ係数であって,係数列の長さが $\Ntp\in\naturalNumbers$ であるとする。
                \begin{align*}
                    \hes(m) &= \begin{cases}
                        \hes(\Ntp-1-m) & \parens*{m =0,1,\dots,(\Ntp-1)/2} \\
                        0 & (\text{otherwise})
                    \end{cases} \\
                    \hos(m) &= \begin{cases}
                        -\hos(\Ntp-1-m) & \parens*{m =0,1,\dots,(\Ntp-1)/2} \\
                        0 & (\text{otherwise})
                    \end{cases}
                \end{align*}
                (上の条件から $\hos((\Ntp-1)/2) = 0$ である。)
                \par
                $\Hes,\;\Hos$ をそれぞれ $\hes,\;\hos$ に対応する周波数スペクトラムとする(但し引数は正規化角周波数)と,簡単な計算を経て次式を得る。
                \begin{align*}
                    \Hes(\Omega) &= \exp\parens*{-i\frac{\Ntp-1}{2}\Omega}\bracks*{\hes\parens*{\frac{\Ntp-1}{2}} + 2\sum_{m=0}^{(\Ntp-1)/2-1}{\hes(m)\cos\Omega\parens*{\frac{\Ntp-1}{2}-m}}} \\
                    \Hos(\Omega) &= 2i\exp\parens*{-i\frac{\Ntp-1}{2}\Omega}\sum_{m=0}^{(\Ntp-1)/2-1}{\hos(m)\sin\Omega\parens*{\frac{\Ntp-1}{2}-m}}
                \end{align*}
                両者とも線形位相特性である(その和については一般にそうならない)。
                さらに $\Hos$ は $\Hes$ に比べて位相が $\pi/2$ だけ進んでいる。
                $\exp\parens*{-i\frac{\Ntp-1}{2}\Omega}$ は因果律を満たすためにフィルタ係数の 0 番目の添え字の位置を時刻の基準にとった為の位相の線形な遅れである。
                後述する Remez のアルゴリズムはこのような線形な位相遅れの部分が取り除かれた実数値関数を引数として受け取るため,適切に配慮すれば問題ない。
                この位相遅れの部分を無視すれば $\Hes$ は偶対称,$\Hos$ は奇対称である。
            \paragraph{Remez のアルゴリズムによる係数の決定}
                \label{Remez のアルゴリズムによる係数の決定}
                $\Hid$ の実部と虚部をそれぞれ $\HidR,\;\HidI$ とする。
                $\Hid$ が Hermite 対称であるから $\HidR$ は偶対称,$i\HidI$ は奇対称である。
                \ref{偶、奇対称の係数がもたらす周波数スペクトラム} の考察に従って $\HidR,\;i\HidI$ をそれぞれ近似する偶対称と奇対称の係数を求めればよいと解る。
                \par
                \cite{DSP_JL} の \inlineCode{remez} 関数は近似対象の周波数スペクトラムとして偶対称と奇対称の実数値関数の両方を扱える(\inlineCode{neg} オプションで選択)。
                さらにその関数をユーザが定義した関数オブジェクトで指定できる。
                実装の都合上,奇対称の周波数スペクトラムを与えて得られる出力は実数係数であるが,これを DTFT すると,偶対称の周波数スペクトラムから得られる出力に比べて位相が $\pi/2$ 進んだ結果になる。
                つまり \inlineCode{remez} 関数に $\HidI$ を与えて得られる係数の DTFT が近似するのは $\HidI$ ではなく $i\HidI$ である。
                \par
                上記の \inlineCode{remez} 関数を使って $\HidR,\;i\HidI$ を近似する係数 $\hesOpt,\;\hosOpt:\integers\to\realNumbers$ が求まる。
                両者の和 $\hopt \coloneq \hesOpt + \hosOpt$ が所望の係数である。
                \par
                $\hesOpt,\;\hosOpt,\;\hopt$ に対応する周波数スペクトラムをそれぞれ $\HesOpt,\;\HosOpt,\;\Hopt$ とする($\HosOpt(\Omega)\in i\realNumbers$ に注意)。
                $\hesOpt,\;\hosOpt,\;\hopt$ の近似誤差,即ち次式で表される $L^\infty$-ノルム
                \begin{align*}
                    \eEs &\coloneq \Verts{\HesOpt - \HidR}_\infty = \max_{\Omega\in[-\pi,\pi]}\abs{\HesOpt(\Omega) - \HidR(\Omega)} \\
                    \eOs &\coloneq \Verts{\HosOpt - i\HidI}_\infty = \max_{\Omega\in[-\pi,\pi]}\abs{\HosOpt(\Omega) - i\HidI(\Omega)} \\
                    \eOpt &\coloneq \Verts{\Hopt - \Hid}_\infty = \max_{\Omega\in[-\pi,\pi]}\abs{\Hopt(\Omega ) -\Hid(\Omega)}
                \end{align*}
                に三角不等式を適用して次式を得る。
                \[ \eOpt = \Verts{\Hopt - \Hid}_\infty = \Verts{\HesOpt - \HidR + (\HosOpt - i\HidI)}_\infty \leq \eEs + \eOs \]
                Remez のアルゴリズム 1 回分の誤差の高々 2 倍程度に収まる。
        \subsubsection{信号が複素数の場合}
            フィルタ係数も複素数であるとする。
            このときは $\Hid$ は Hermite 対称とは限らないが,次のように Hermite 対称、歪 Hermite 対称 (Skew-Hermitian) な 2 つの関数 $\HidH,\;\HidSH$ の和に分解できる。
            \begin{gather*}
                \Hid(\Omega) = \HidH(\Omega) + \HidSH(\Omega) \\
                \text{where}\quad\HidH(\Omega)\coloneq\frac{1}{2}\parens*{\Hid(\Omega) + \conj{\Hid(-\Omega)}},\quad\HidSH(\Omega)\coloneq\frac{1}{2}\parens*{\Hid(\Omega) - \conj{\Hid(-\Omega)}}
            \end{gather*}
            $\HidH$ の実部と虚部をそれぞれ $\HidHr,\;\HidHi$ とすると前者は偶対称,後者は奇対称である。
            また $\HidSH$ の実部と虚部をそれぞれ $\HidSHr,\;\HidSHi$ とすると前者は奇対称,後者は偶対称である。
            これら 4 つに対して Remez のアルゴリズムを使って,近似するためのフィルタ係数(実数値)を求めて(適宜,虚数単位を掛けながら)和をとれば所望の係数となる。
            \par
            設計するフィルタ係数を $h:\integers\to\realNumbers$ とし,それらの実部と虚部をそれぞれ $\hr,\;\hi$ とする。
            $\hr$ によって $\HidH$ を近似し,$\hi$ によって $\HidSH$ を近似する。
            \par
            $\hr$ を偶対称、奇対称な 2 つの成分に分解し,それぞれを $\hrEs,\;\hrOs$ とする。
            この分解は常に可能である。具体的には $\hr$ とその左右反転の和の 1/2 倍を $\hrEs$ とし,$\hr$ とその左右反転の差の 1/2 倍を $\hrOs$ とすればよい。
            $\hrEs,\;\hrOs$ については \ref{Remez のアルゴリズムによる係数の決定} と同じ方法で最適値 $\hrEsOpt,\;\hrOsOpt$ が求まる。
            \par
            次に $\hi$ の最適値を求める。
            $\hi$ を偶対称、奇対称な 2 つの成分に分解し,それぞれを $\hiEs,\;\hiOs$ とする。
            天下り的であるが(妥当性は以降の解説で直ぐに解る) Remez のアルゴリズムを使って $\hiEs$ によって $\HidSHi$ を近似して最適値を $\hiEsOpt$ とし,$\hiOs$ によって $-i\HidSHr$ を近似して最適値を $\hiOsOpt$ とする。
            \par
            $\hrEsOpt,\;\hrOsOpt,\;\hiEsOpt,\;\hiOsOpt$ に対応する周波数スペクトラムを $\HrEsOpt,\;\HrOsOpt,\;\HiEsOpt,\;\HiOsOpt$ とする($\HrOsOpt(\Omega),\;\HiOsOpt(\Omega)\in i\realNumbers$ に注意)。
            最終的な最適係数は $\hopt = \hrEsOpt + \hrOsOpt + i(\hiEsOpt + \hiOsOpt)$ である。
            それに対応する周波数スペクトラム $\Hopt$ は次式である。
            但し因果律を満たすために生じる位相遅れ $\exp\parens*{-i\frac{\Ntp-1}{2}\Omega}$ は無害であるから無視する。
            \begin{align*}
                \Hopt &= \HrEsOpt + \HrOsOpt + i(\HiEsOpt + \HiOsOpt) = \HrEsOpt + \HrOsOpt + i\HiEsOpt + i\HiOsOpt \\
                &\approx \HidHr + i\HidHi + i\HidSHi + \HidSHr = \Hid
            \end{align*}
            $\hrEsOpt,\;\hrOsOpt,\hiEsOpt,\;\hiOsOpt,\;\hopt$ の近似誤差,即ち次式で表される $L^\infty$-ノルム
            \begin{align*}
                \erEs &\coloneq \Verts{\HrEsOpt - \HidHr}_\infty = \max_{\Omega\in[-\pi,\pi]}\abs{\HrEsOpt(\Omega) - \HidHr(\Omega)} \\
                \erOs &\coloneq \Verts{\HrOsOpt - i\HidHi}_\infty = \max_{\Omega\in[-\pi,\pi]}\abs{\HrOsOpt(\Omega) - i\HidHi(\Omega)} \\
                \eiEs &\coloneq \Verts{\HiEsOpt - \HidSHi}_\infty = \max_{\Omega\in[-\pi,\pi]}\abs{\HiEsOpt(\Omega) - \HidSHi(\Omega)} \\
                \eiOs &\coloneq \Verts{\HiOsOpt + i\HidSHr}_\infty = \max_{\Omega\in[-\pi,\pi]}\abs{\HiOsOpt(\Omega) + i\HidSHr(\Omega)} \\
                \eOpt &\coloneq \Verts{\Hopt - \Hid}_\infty = \max_{\Omega\in[-\pi,\pi]}\abs{\Hopt(\Omega ) -\Hid(\Omega)}
            \end{align*}
            に三角不等式を適用して次式を得る。
            \begin{align*}
                \eOpt &= \Verts{\Hopt - \Hid}_\infty \\
                &= \Verts{\HrEsOpt - \HidHr + (\HrOsOpt - i\HidHi) + i(\HiEsOpt - \HidSHi) + (i\HiOsOpt - \HidSHr)}_\infty \\
                &\leq \erEs + \erOs + \eiEs + \eiOs
            \end{align*}
            Remez のアルゴリズム 1 回分の誤差の高々 4 倍程度に収まる。
