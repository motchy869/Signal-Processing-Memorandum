\part{Z 変換}
	\chapter{両側 Z 変換}
		\section{逆 Z 変換}
			\label{inv_Z_trans}
			\begin{shadebox}
				$X:\complexNumbers\to\complexNumbers$ は離散時間信号 $x:\integers\to\complexNumbers$ の Z 変換であるとする。
				このとき次式が成り立つ。
				\begin{equation}
					\label{equation:inv_Z_trans}
					x(n) = \frac{1}{2\pi i}\oint_C X(z)z^{n-1}\mathrm{d}z
				\end{equation}
				ここに積分路 $C$ は $X(z)$ のすべての極を含む反時計回りの単純閉曲線である。
			\end{shadebox}
			\begin{proof}
				\begin{align*}
					\frac{1}{2\pi i}\oint_C X(z)z^{n-1}\mathrm{d}z &= \frac{1}{2\pi i}\oint_C \parens*{\sum_{k=-\infty}^\infty x(k)z^{-k}}z^{n-1}\mathrm{d}z \\
					&= \sum_{k=-\infty}^\infty x(k)\frac{1}{2\pi i}\oint_C z^{n-k-1}\mathrm{d}z = \sum_{k=-\infty}^\infty x(k)\delta_{n,k} = x(n)
				\end{align*}
			\end{proof}
		\section{最終値定理}
			\begin{shadebox}
				$X(z)\;(z\in\complexNumbers)$を離散時間信号$x(n)\;(n \in \integers,\;\forall n<0,x(n)=0)$のZ変換とする。
				$\lim_{n\to\infty} x(n)$が存在するとき次が成り立つ。
				\[ \lim_{z\to1}(z-1)F(z) = \lim_{n\to\infty} x(n) \]
				但し上式に於ける$\lim_{z\to1}$では$z$が実軸上で右側から1に近づくことを意味する。
			\end{shadebox}
			\begin{proof}
				\quad\par
				$\alpha = \lim_{n\to\infty} x(n)$とする。
				発想としては,十分大きい$N\in\naturalNumbers$に対して$\sum_{k=N+1}^\infty x(k)z^{-k} \sim \sum_{k=N+1}^\infty \alpha z^{-k} = \alpha z^{-N}\frac{1}{z-1}$となることを利用する。
				\par
				任意の$\varepsilon \in (0,1)$に対してある$N\in\naturalNumbers$が存在して$\forall n\geq N,\;|x(n)-\alpha|<\varepsilon$となる。
				\begin{align*}
					\quad &\lim_{z\to1}(z-1)F(z) - \alpha = \lim_{z\to1}(z-1)z^N F(z) - \alpha \\
					&= \lim_{z\to1}(z-1)z^N\left(\sum_{k=0}^{N-1} x(k)z^{-k} + \sum_{k=N+1}^\infty x(k)z^{-k}\right) - (z-1)z^N\sum_{k=N+1}^\infty \alpha z^{-k} \\
					&= \lim_{z\to1}(z-1)z^N \sum_{k=N+1}^\infty (x(n) - \alpha)z^{-k} \quad \left(\sum_{k=0}^{N-1}\text{の項は極限で消える}\right) \tag{1} \\
					|(1)| &\leq \lim_{z\to1}(z-1)z^N \sum_{k=N+1}^\infty |x(n) - \alpha|z^{-k} < \lim_{z\to1}(z-1)z^N \sum_{k=N+1}^\infty \varepsilon z^{-k} = \varepsilon
				\end{align*}
			\end{proof}
		\section{複素指数関数入力に対する伝達関数の作用}
			\begin{shadebox}
				$A>0,\;\omega \in \realNumbers$とする。
				離散時間信号$x: \realNumbers \to \complexNumbers$を次のように定める。
				\[
					x(n) =
					\begin{cases}
						Ae^{i\Omega n} & (n\geq 0) \\
						0 & (n<0)
					\end{cases}
				\]
				$H: z\in\complexNumbers \mapsto H(z) \in \complexNumbers$を,$1/z$を変数とした有理式として既約であるような有理関数とする。
				また,$H$の極の絶対値は全て1未満であるとする。
				伝達関数が$H(z)$である離散時間システムに信号$x$を入力した時の出力を$y$とすると,十分大きい$n$に対して
				$y(n) \sim H(\NapierE^{i\Omega})x(n)$となる。
			\end{shadebox}
			\begin{proof}
				\quad\par
				$N_\text{p}$を$H(s)$の相異なる極の個数とし,それら極を$p_0,\dots,p_{N_\text{p}}$とする。
				極$p_k$の次数を$N_{\text{p},k}$とし,$H(z)$の部分分数展開を
				\[ H(z) = c_0 + \sum_{k=1}^{N_\mathrm{p}} \sum_{l=1}^{N_{\mathrm{p},k}} \frac{c_{k,l}}{(1-p_kz^{-1})^l} \]
				とする。
				ここに$c_0,c_{k,l}\;(k=1,\dots,N_\mathrm{p},l=1,\dots,N_{\mathrm{p},k})$は適当な複素数である。
				$x,y$のZ変換をそれぞれ$X,Y$とすると$Y(z) = H(z)F(z) = A H(z)/(1-\NapierE^{i\Omega}z^{-1})$である。
				これの部分分数展開に現れる,$1/(1-p_k z^{-1})^l\;(k=1,\dots,N_\mathrm{p},l=1,\dots,N_{\mathrm{p},k})$に比例する項は逆Z変換すると$n$の多項式と公比$p_k$の等比級数の積となり,$n\to\infty$で0に収束する。
				(このことはZ変換の性質: 時間シフト$\mathcal{Z}[x(n+k)] = z^kX(z)$,およびZ領域微分$\mathcal{Z}[nx(n)] = -z\derivLong{\mathcal{Z}[x(n)]}{z}{}$を繰り返し用いることで分かる)
				\par
				残りの項,すなわち$1/(1-\NapierE^{i\Omega}z^{-1})$に比例する項は$AH(\NapierE^{i\Omega})/(1-\NapierE^{i\Omega}z^{-1}) = H(\NapierE^{i\Omega})X(z)$となる。
			\end{proof}
	\chapter{片側 Z 変換}
		\section{信号とその片側 Z 変換の一対一対応}
			離散時間信号 $x:\integers\to\complexNumbers$ とその片側 Z 変換 $X$ は非負の時間領域について一対一対応する。
			このことは \ref{inv_Z_trans} の証明から明らかである。
			片側 Z 変換の結果に対して式 \eqref{equation:inv_Z_trans} に於いて負の時刻を指定すると,結果は 0 である。
		\section{畳み込みの片側 Z 変換}
			\label{single_sided_Z_transform_of_convolution}
			\begin{shadebox}
				離散時間信号 $x,y:\integers\to\complexNumbers$ がともに負の時刻について 0 であるとし,それぞれの片側 Z 変換を $X,Y$ とする。
				次式が成り立つ。
				\[ \SSZTrans{x*y}(z) = X(z)Y(z) \]
			\end{shadebox}
			\begin{proof}
				\begin{align*}
					\SSZTrans{x*y}(z) &= \sum_{k=0}^\infty \sum_{l=-\infty}^\infty x(l)y(k-l) z^{-k} = \sum_{k=0}^\infty \sum_{l=0}^\infty x(l)y(k-l) z^{-k} \\
					&= \sum_{l=0}^\infty x(l)z^{-l} \sum_{k=0}^\infty y(k-l) z^{-(k-l)} = \sum_{l=0}^\infty x(l)z^{-l} \sum_{k=l}^\infty y(k-l) z^{-(k-l)} \\
					&= \sum_{l=0}^\infty x(l)z^{-l} Y(z) = X(z)Y(z)
				\end{align*}
			\end{proof}
