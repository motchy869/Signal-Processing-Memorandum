\chapter{信号検出}
    \section{位置特定に於けるcos類似度による方法と最良近似による方法の等価性}
        複素数列で表される受信信号$\{s_i\}$の中から特定のパターン(「参照信号」と呼ぶ)を見つけ出したい時がある。
        例えば無線通信に於いては送信機から「同期ワード (Sync Word, SW)」と呼ばれる数十bit分の変調信号が一定周期で送出されており、これが「フレーム」と呼ばれる単位の区切り位置の決定に使われる。
        受信機は常にSWを探索し、フレームの区切り位置を絶えずトラッキングする必要がある。
        なぜならば、送信機,受信機に搭載されているクロック発生器には僅かだが誤差があり、受信機から見た送信機の送出する信号の時間軸は少しずつズレていくからである。
        \par
        今、受信信号列の全体的な位相には関心が無いものとする。
        つまり、信号全体に大きさ1の複素定数を乗算する操作は受信側の信号処理にとって影響がないものとする。
        現実の無線機で言えば、例えば$\pi/4$シフトQPSKがそうである。
        \par
        受信信号から参照信号を検出する方法として、直観的に次の2つの方法を思いつくだろう。
        \subsection{手法1: cos類似度の絶対値の最大化}
            \label{手法1: cos類似度の絶対値の最大化}
            参照信号の長さを$L\in\naturalNumbers$, 参照信号を$\bm{d} \in \complexNumbers^L$, 受信信号中のテスト領域を$\bm{s}^{(i)} \coloneqq [s_i,s_{i+1},\dots,s_{i+L-1}]^\top \in \complexNumbers^L$とするとき、$\bm{d}$と$\bm{s}^{(i)}$の$\cos$類似度の複素数版
            \[ \frac{\bm{d}^*\bm{s}^{(i)}}{\norm{\bm{d}}_2\norm{\bm{s}^{(i)}}_2} \]
            の位相を無視し、絶対値の2乗(2乗を使うのは、平方根の計算を無くして計算量を抑える為)で評価する。
            $\norm{\bm{d}}_2$は$\bm{s}^{(i)}$に依存しないので評価値同士の大小比較に必要ないから取り除く。
            すると評価関数$c$として次式を得る。
            \[ c(i) = \frac{|\bm{d}^*\bm{s}^{(i)}|^2}{\norm{\bm{s}^{(i)}}_2^2} \]
            これが最大となる$i$を参照信号の存在位置と見做す。
        \subsection{手法2: 最良近似}
            \ref{手法1: cos類似度の絶対値の最大化}で定義した記号をここでも用いる。
            受信信号中の参照信号は「参照信号+ゲイン変化+位相回転+ノイズ」の形で存在している。
            そこで、参照信号に定数$\alpha$を掛けて$\bm{s}^{(i)}$との差を取った絶対値の2乗を参照信号のL-2ノルムの2乗で正規化した値が最小となるように$\alpha$を選び、そのときの差の絶対値の2乗が最小になるような位置をもって参照信号の存在位置と見做す。
            評価関数$\tilde{c}$は次式である。
            \[ \tilde{c}(i) = \frac{1}{\norm{\bm{s}^{(i)}}_2^2}\min_{\alpha \in \complexNumbers}\norm{\alpha\bm{d} - \bm{s}^{(i)}}_2^2 \]
            正規化する理由は、テスト領域の強度の影響を減らすためである。
            テスト領域の形が参照信号と大きく異なっていても、テスト領域の強度が小さければ$\min_{\alpha \in \complexNumbers}\norm{\alpha\bm{d} - \bm{s}^{(i)}}_2^2$は小さくなり、誤った推定結果を導き得る。
            上の最小化問題の解は解析的に求められる。
            $f(\alpha) \coloneqq \norm{\alpha\bm{d} - \bm{s}^{(i)}}_2^2$について微小な$\Delta\alpha$を考え、$f(\alpha+\Delta\alpha) - f(\alpha)$の変化量の$\Delta\alpha$の1次の項が0になるような$\mathring{\alpha}$が解である。これは次式である。
            \[
                \mathring{\alpha} = \frac{\bm{d}^*\bm{s}^{(i)}}{\norm{\bm{d}}_2^2}
            \]
            よって$\tilde{c}(i)$は次式である。
            \[ \tilde{c}(i) = \frac{1}{\norm{\bm{s}^{(i)}}_2^2}\norm{\bm{s}^{(i)} - \frac{\bm{d}^*\bm{s}^{(i)}}{\norm{\bm{d}}_2^2}\bm{d}}_2^2 \]
        \subsection{手法1,2の等価性}
            実は手法1と2は等価である。
            すなわち次の命題は真である。
            \[ \frac{1}{\norm{\bm{s}^{(i)}}_2^2}\norm{\bm{s}^{(i)} - \frac{\bm{d}^*\bm{s}^{(i)}}{\norm{\bm{d}}_2^2}\bm{d}}_2^2 < \frac{1}{\norm{\bm{s}^{(j)}}_2^2}\norm{\bm{s}^{(j)} - \frac{\bm{d}^*\bm{s}^{(j)}}{\norm{\bm{d}}_2^2}\bm{d}}_2^2 \iff \frac{|\bm{d}^*\bm{s}^{(i)}|^2}{\norm{\bm{s}^{(i)}}_2^2} > \frac{|\bm{d}^*\bm{s}^{(j)}|^2}{\norm{\bm{s}^{(j)}}_2^2} \]
            これを示す。
            \begin{align*}
                \frac{1}{\norm{\bm{s}^{(i)}}_2^2}\norm{\bm{s}^{(i)} - \frac{\bm{d}^*\bm{s}^{(i)}}{\norm{\bm{d}}_2^2}\bm{d}}_2^2 &= \frac{1}{\norm{\bm{s}^{(i)}}_2^2}\left[ \norm{\bm{s}^{(i)}}_2^2 + \frac{|\bm{d}^*\bm{s}^{(i)}|^2}{\norm{\bm{d}}_2^4}\norm{\bm{d}}_2^2 - \frac{\bm{d}^*\bm{s}^{(i)}}{\norm{\bm{d}}_2^2}{\bm{s}^{(i)}}^*\bm{d} - \frac{\conj{\bm{d}^*\bm{s}^{(i)}}}{\norm{\bm{d}}_2^2}\bm{d}^*\bm{s}^{(i)} \right] \\
                &= \frac{1}{\norm{\bm{s}^{(i)}}_2^2}\left[ \norm{\bm{s}^{(i)}}_2^2 + \frac{|\bm{d}^*\bm{s}^{(i)}|^2}{\norm{\bm{d}}_2^2} - 2\frac{|\bm{d}^*\bm{s}^{(i)}|^2}{\norm{\bm{d}}_2^2} \right] \\
                &= 1 - \frac{|\bm{d}^*\bm{s}^{(i)}|^2}{\norm{\bm{d}}_2^2}
            \end{align*}
            であり、
            \[ 1 - \frac{|\bm{d}^*\bm{s}^{(i)}|^2}{\norm{\bm{d}}_2^2} < 1 - \frac{|\bm{d}^*\bm{s}^{(j)}|^2}{\norm{\bm{d}}_2^2} \iff \frac{|\bm{d}^*\bm{s}^{(i)}|^2}{\norm{\bm{s}^{(i)}}_2^2} > \frac{|\bm{d}^*\bm{s}^{(j)}|^2}{\norm{\bm{s}^{(j)}}_2^2} \]
            であることから命題が真であることがわかる。