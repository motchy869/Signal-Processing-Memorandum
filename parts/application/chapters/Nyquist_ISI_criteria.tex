\chapter{Nyquist ISI 基準}
    これは大雑把に言うと Fourier 変換が存在する連続時間信号 $h:\realNumbers\to\complexNumbers$ が 1 つの例外の時刻を除いて、ある周期 $\Ts>0$ (s は symbol の意味)の整数倍の時刻で 0 になるための必要十分条件である。
    限定された周波数帯域を使って通信する際に受信側で情報を正しく復元するために重要な性質であり、詳細は \cite{Nyquist_ISI_crit} にある。
    数式で表すと次である。
    \[
        h(n\Ts) = \begin{cases}
            1 & n=0 \\
            0 & n\in\integers\setminus\{0\}
        \end{cases}
        \iff \forall f\in\realNumbers,\;\frac{1}{\Ts}\sum_{n=-\infty}^\infty H(f-n/\Ts) = 1
    \]
    ここに $H$ は $h$ の Fourier 変換である。
    \cite{Nyquist_ISI_crit} には $\Rightarrow$ の証明のみがある。
    本書では $\Leftarrow$ を証明する。
    \begin{proof}
        \begin{align*}
            1 &= \frac{1}{\Ts}\sum_{n=-\infty}^\infty H(f-n/\Ts) = \frac{1}{\Ts}\sum_{n=-\infty}^\infty\integrate{-\infty}{\infty}{h(t)\exp\parens*{-i2\pi(f-n/\Ts)t}}{}{t} \\
            &= \frac{1}{\Ts}\integrate{-\infty}{\infty}{h(t)\exp(-i2\pi ft)\sum_{n=-\infty}^\infty\exp\parens*{i2\pi nt/\Ts}}{}{t} \tag{1}
        \end{align*}
        ここで次の関係式を使う(Dirac のデルタ関数の無限和に関する頻出の関係式であり、類似の形の式が以前の部や章で使われている)。
        \[ \sum_{n=-\infty}^\infty\exp\parens*{i2\pi nt/\Ts} = 2\pi\Ts\sum_{n=-\infty}^\infty\delta(2\pi t-2\pi\Ts n) = \Ts\sum_{n=-\infty}^\infty\delta(t-n\Ts) \]
        これを式 (1) に適用して次式を得る。
        \begin{align*}
            1 &= \integrate{-\infty}{\infty}{h(t)\exp\parens*{-i2\pi ft}\sum_{n=-\infty}^\infty\delta(t-n\Ts)}{}{t} = \sum_{n=-\infty}^\infty\integrate{-\infty}{\infty}{h(t)\exp\parens*{-i2\pi ft}\delta(t-n\Ts)}{}{t} \\
            &= \sum_{n=-\infty}^\infty h(n\Ts)\exp\parens*{-i2\pi fn\Ts}
        \end{align*}
        右辺は $f$ に関する周期 $1/\Ts$ の関数の Fourier 級数であり、$h(n\Ts)$ は Fourier 係数である。
        左辺が 1 であることから $h(0) = 1,\;h(n\Ts)\;(n\neq 0) = 0$ である(より丁寧に論じるなら、前記の式の両辺に $\exp(i2\pi fk\Ts)\;(k\in\integers)$ を掛けて区間 $[-1/(2\Ts),1/(2\Ts)]$ で積分する。その結果が $k$ にどう依存するかを調べる)。
    \end{proof}