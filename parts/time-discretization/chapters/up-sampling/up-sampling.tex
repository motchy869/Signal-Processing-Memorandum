\chapter{アップサンプリング}
    \section{アップサンプリングされた信号の周波数特性}
        \newcommand{\xda}{x_{\text{d},1}}
        \newcommand{\Xda}{X_{\text{d},1}}
        \newcommand{\xdb}{x_{\text{d},2}}
        \subsection{動機}
            離散時間信号をアップサンプリングしてDACで出力したときの周波数スペクトルを計算したい。
            DACの量子化誤差は無視する。
        \subsection{主張}
            記号を次のように定義する。
            \begin{itemize}
                \item $R\in\naturalNumbers,\;R\geq 2$ : アップサンプリングレート
                \item $\xda:\integers\to\complexNumbers$ : 離散時間信号
                \item $\Ts>0$ : $\xda$ のサンプル周期
                \item $\Xda$ : $\xda$ のDTFT
                \item $\xdb$ : $\xda$ を $R$ 倍にアップサンプリング(元の信号のサンプル同士の間に $R-1$ 個の0を追加)した離散時間信号
                \item $x_1$ : $\Ts$ をサンプル周期として $\xda$ の0次ホールドで生成した階段状の連続時間信号
                \item $x_2$ : $\Ts/R$ をサンプル周期として $\xdb$ の0次ホールドで生成した階段状の連続時間信号
            \end{itemize}
            $u_1:\realNumbers\to\braces{0,1}$ を幅 $\Ts$ のパルスとする。
            \[
                u_1(t) = \begin{cases}
                    1 & 0\leq t < \Ts \\
                    0 & \text{otherwise}
                \end{cases}
            \]
            $u_2:\realNumbers\to\braces{0,1}$ を幅 $\Ts/R$ のパルスとする。
            \[
                u_2(t) = \begin{cases}
                    1 & 0\leq t < \Ts/R \\
                    0 & \text{otherwise}
                \end{cases}
            \]
            $x_1$ は次式で表される。
            \[ x_1(t) = \sum_{n=-\infty}^\infty \xda(n)u_1(t-n\Ts) \]
            $x_2$ は次式で表される。
            \[ x_2(t) = \sum_{n=-\infty}^\infty \xdb(n)u_2(t-n\Ts/R) = \sum_{m=-\infty}^\infty \xdb(R m)u_2(t-m\Ts) = \sum_{n=-\infty}^\infty \xda(n)u_2(t-n\Ts) \]
            次の図は $\Ts=1,R=4,\xda(n) = \sin\parens{2\pi*n/12}\;(0\leq n\leq 24),\;\xda(n) = 0\;(n<0,24<n)$ の例である。
            \begin{figure}[H]
                \centering
                \includegraphics[keepaspectratio, scale=0.8]
                {\currfiledir/imgs/x1,x2.pdf}
                \caption{$x_1,x_2$の例}
                \label{アップサンプリング前後のDAC出力の例}
            \end{figure}
            以上の下、 $x_2$ のFourier変換 $X_2$ は次式である。
            \[ X_2(\omega) = \frac{\Ts}{R\sqrt{2\pi}}\exp\parens*{-i\frac{\Ts}{2R}\omega}\parens*{\sinc \frac{\Ts}{2R}\omega}\Xda(\omega) \]
        \subsection{導出}
            \begin{proof}
                \quad\par
                \[ X_2(\omega) = \FT{\sum_{n=-\infty}^\infty \xda(n)u_2(t-n\Ts)}(\omega) = \sum_{n=-\infty}^\infty \xda(n)\FT{u_2(t-n\Ts)}(\omega) \]
                ここで\ref{0次ホールドされた離散時間信号の周波数特性}と同様にして次式が成り立つ。
                \[ \FT{u_2(t-n\Ts)}(\omega) = \frac{\Ts}{R\sqrt{2\pi}}\exp\parens*{-i \omega n\Ts}\exp\parens*{-i\frac{\omega\Ts}{2R}}\sinc\frac{\omega\Ts}{2R} \]
                よって次式が成り立つ。
                \begin{align*}
                    X_2(\omega) &= \sum_{n=-\infty}^\infty \xda(n)\frac{\Ts}{R\sqrt{2\pi}}\exp\parens*{-i \omega n\Ts}\exp\parens*{-i\frac{\omega\Ts}{2R}}\sinc\frac{\omega\Ts}{2R} \\
                    &= \frac{\Ts}{R\sqrt{2\pi}}\exp\parens*{-i\frac{\omega\Ts}{2R}}\parens*{\sinc\frac{\omega\Ts}{2R}} \sum_{n=-\infty}^\infty \xda(n)\exp\parens*{-i \omega n\Ts} \\
                    &= \frac{\Ts}{R\sqrt{2\pi}}\exp\parens*{-i\frac{\omega\Ts}{2R}}\parens*{\sinc\frac{\omega\Ts}{2R}} \Xda(\omega) \tag{a}
                \end{align*}
            \end{proof}
        \subsection{考察}
            アップサンプリング前の信号 $x_1$ については\ref{0次ホールドされた離散時間信号の周波数特性}より、そのFourier変換は次式である。
            \[ X_1(\omega) = \frac{\Ts}{\sqrt{2\pi}}\exp\parens*{-i\frac{\Ts}{2}\omega}\parens*{\sinc \frac{\Ts}{2}\omega}\Xda(\omega) \tag{b} \]
            式(a),(b)を見比べると $\Xda$ を共通して含んでおり、それ以外の箇所でアップサンプリングにより $\Ts$ が $\Ts/R$ に置き換わっていることがわかる。
            このことから、アップサンプリングにより高調波の位置は変わらず、広域の減衰や位相回転が緩やかになることがわかる。
            次の図は\ref{アップサンプリング前後のDAC出力の例}に対応するDTFTとFourier変換の絶対値の例である。
            \begin{figure}[H]
                \centering
                \includegraphics[keepaspectratio, scale=0.8]
                {\currfiledir/imgs/Xd1.pdf}
                \caption{$\Xda$の例。横軸は正規化各周波数。}
            \end{figure}
            \begin{figure}[H]
                \centering
                \includegraphics[keepaspectratio, scale=0.8]
                {\currfiledir/imgs/FT_of_x1,x2.pdf}
                \caption{$X_1,X_2$の例。横軸は正規化各周波数。}
            \end{figure}