\chapter{積分の離散近似}
    \section{Euler法}
        \subsection{背景}
            物理系をディジタル計算機で制御するにあたり、積分をEuler法で近似することがある。
            本節では正弦波をEuler法で近似的に積分した際の出力の窓関数付きFourier変換を導出し、高周波領域での位相変化、エイリアシングについて考察する。
        \subsection{導出}
            \newcommand{\xdd}{x_\text{dd}}
            $f_0>0$とし、連続時間の複素正弦波信号$u:t\in\realNumbers\mapsto\exp(i 2\pi f_0 t)$を考える。
            これを時刻$0$から$t\geq 0$まで積分した信号は$v(t) = \bigl(\exp(i 2\pi f_0 t)-1\bigr) / (i 2\pi f_0)$である。
            \ref{0次ホールド機構の周波数特性}と同様に、矩形窓を通した、周波数表示された$v$のFourier変換を考える(窓の幅をサンプリング周期の整数倍に限っても影響が少ないことの説明は\ref{0次ホールド機構の周波数特性}で述べられている)。
            $N\in\naturalNumbers$とし、窓の幅を$T=N\Tsamp$とする。
            $v$の窓付きFourier変換を窓の幅で規格化したものは次式である。
            但し計算は容易なので過程は省略した。
            \begin{align*}
                &\phantom{=} V(f) = \frac{1}{T} \integrate{0}{T}{v(t)\exp(-i 2\pi f t)}{}{t} \\
                &= \frac{1}{i 2\pi f_0 T} \left\{\frac{1}{i 2\pi (f-f_0)}\bigl[1 - \exp\bigl(-i 2\pi (f-f_0)T\bigr)\bigr] + \frac{1}{i 2\pi f}\bigl(\exp(-i 2\pi f T) - 1\bigr)\right\}
            \end{align*}
            \par
            次に、$u$の積分をサンプリング周期$\Tsamp>0$のEuler法で近似したものを考える。
            Euler法で積分した結果の離散時間信号を$\xdd:\integers\to\complexNumbers$とすると、これは漸化式$\xdd(n) = \xdd(n-1) + \Tsamp u\bigl((n-1)\Tsamp\bigr)$に従う。
            但し初期条件として$\xdd(0)=0$とする。
            この漸化式を解き、次式を得る。
            \[ \xdd(n) = \Tsamp\frac{1-\exp(i 2\pi f_0 n\Tsamp)}{1-\exp(i 2\pi f_0\Tsamp)} \]
            これを0次ホールドして得られる連続時間信号を$\xd(t) \coloneqq \xdd(\floor{t/\Tsamp}\Tsamp)$とする。
            先ほど$v$に対して行ったのと同様に窓付きFourier変換$\Xd$を計算すると、次式を得る。
            但し計算は容易なので過程の多くを省略した。
            \begin{align*}
                &\phantom{=} \Xd(f) = \frac{1}{T} \integrate{0}{T}{\xd(t)\exp(-i 2\pi f t)}{}{t} = \sum_{k=0}^{N-1} \frac{1}{T} \integrate{k\Tsamp}{(k+1)\Tsamp}{\xd(t)\exp(-i 2\pi f t)}{}{t} \\
                &= \frac{1}{i 2\pi f N}\times\frac{1-\exp(-i 2\pi f\Tsamp)}{1-\exp(i 2\pi f_0\Tsamp)} \left\{\frac{1-\exp(-i 2\pi f \Tsamp N)}{1-\exp(-i 2\pi f \Tsamp)} - \frac{1-\exp(-i 2\pi (f-f_0) \Tsamp N)}{1-\exp(-i 2\pi (f-f_0) \Tsamp)}\right\}
            \end{align*}
        \subsection{数値例}
