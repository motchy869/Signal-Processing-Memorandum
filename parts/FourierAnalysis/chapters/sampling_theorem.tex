\chapter{サンプリング定理}
    \begin{shadebox}
        $d\in\naturalNumbers,\;W_l>0\;(l=1,2,,\dots,d),\;\Omega\coloneqq\prod_{l=1}^d [-W_l,W_l]$とする。
        $f:\realNumbers^d\to\realNumbers$のFourier変換$\mathcal{F}(f,\bm{\omega})$が存在してその台が$\Omega$に含まれるとき、次式が成り立つ。
        \[ f(\bm{x}) = \sum_{\bm{n}\in\integers^d}f\left(\pi\frac{n_1}{W_1},\dots,\pi\frac{n_d}{W_d}\right)\prod_{l=1}^d\sinc W_l\left(x_l - \pi\frac{n_l}{W_l}\right) \]
        つまり$f$の各点での評価値を沢山集めて$f$を任意の精度で近似できる。
        \par
        角周波数$W_l$のかわりに周波数$F_l=W_l/(2\pi)$を使うと上式は次式になる。
        \[ f(\bm{x}) = \sum_{\bm{n}\in\integers^d}f\left(\frac{n_1}{2F_1},\dots,\frac{n_d}{2F_d}\right)\prod_{l=1}^d\sinc 2\pi F_l\left(x_l - \frac{n_l}{2F_l}\right) \]
    \end{shadebox}
    \begin{proof}
        \quad\par
        $\mathcal{F}(f,\bm{\omega})$の台が超直方体$\Omega$に含まれるから$\mathcal{F}(f,\bm{\omega})$はFourier級数展開できる。
        第$\bm{n}$Fourier係数を$c(\mathcal{F}(f),\bm{n})$とすると
        \[ \mathcal{F}(f,\bm{\omega}) = \sum_{\bm{n}\in\integers^d} c(\mathcal{F}(f),\bm{n}) \exp i\sum_{l=1}^d n_l\frac{\omega_l}{W_l}\pi \]
        となる。$c(\mathcal{F}(f),\bm{n})$は次式で求まる。
        \begin{align*}
            c(\mathcal{F}(f),\bm{n}) &= \left(\prod_{l=1}^d 2W_l\right)^{-1} \integrate{\Omega}{}{\mathcal{F}(f,\bm{\xi}) \exp (-i)\sum_{l=1}^d n_l\frac{\xi_l}{W_l}\pi}{}{\bm{\xi}} \\
            &= (2\pi)^{d/2} \left(\prod_{l=1}^d 2W_l\right)^{-1} (2\pi)^{-d/2} \integrate{\textcolor{darkpastelgreen}{\realNumbers^d}}{}{\mathcal{F}(f,\bm{\xi}) \exp i\sum_{l=1}^d \left(\frac{-n_l}{W_l}\pi\right)\xi_l}{}{\bm{\xi}} \\
            &= (2\pi)^{d/2} \left(\prod_{l=1}^d 2W_l\right)^{-1} \mathcal{F}^{-1}\left(\mathcal{F}(f),-\pi\bm{n}\HadamardDiv\bm{W}\right) \\
            &= (2\pi)^{d/2} \left(\prod_{l=1}^d 2W_l\right)^{-1} f\left(-\pi\bm{n}\HadamardDiv\bm{W}\right)
        \end{align*}
        $f$は$\mathcal{F}(f)$のFourier逆変換で次のようにして求まる。
        \begin{align*}
            f(\bm{x}) &= \mathcal{F}^{-1}\left(\mathcal{F}(f),\bm{x}\right) = (2\pi)^{-d/2} \integrate{\realNumbers^d}{}{\mathcal{F}(f,\bm{\omega})\exp i\bm{\omega}^\top\bm{x}}{}{\bm{\omega}} = (2\pi)^{-d/2} \integrate{\Omega}{}{\mathcal{F}(f,\bm{\omega})\exp i\bm{\omega}^\top\bm{x}}{}{\bm{\omega}} \\
            &= (2\pi)^{-d/2} \integrate{\Omega}{}{\sum_{\bm{n}\in\integers^d} c(\mathcal{F}(f),\bm{n}) \left(\exp i\sum_{l=1}^d n_l\frac{\omega_l}{W_l}\pi\right) \exp i\bm{\omega}^\top\bm{x}}{}{\bm{\omega}} \\
            &= (2\pi)^{-d/2} \sum_{\bm{n}\in\integers^d} \integrate{\Omega}{}{c(\mathcal{F}(f),\bm{n}) \exp i \bm{\omega}^\top\left(\bm{x} + \pi\bm{n}\HadamardDiv\bm{W}\right) }{}{\bm{\omega}} \\
            &= (2\pi)^{-d/2} \sum_{\bm{n}\in\integers^d} \integrate{\Omega}{}{(2\pi)^{d/2} \left(\prod_{l=1}^d 2W_l\right)^{-1} f\left(-\pi\bm{n}\HadamardDiv\bm{W}\right) \exp i \bm{\omega}^\top\left(\bm{x} + \pi\bm{n}\HadamardDiv\bm{W}\right) }{}{\bm{\omega}} \\
            &= \left(\prod_{l=1}^d 2W_l\right)^{-1} \sum_{\bm{n}\in\integers^d} f\left(-\pi\bm{n}\HadamardDiv\bm{W}\right) \integrate{\Omega}{}{\exp i \bm{\omega}^\top\left(\bm{x} + \pi\bm{n}\HadamardDiv\bm{W}\right) }{}{\bm{\omega}}
        \end{align*}
        ここで
        \begin{align*}
            \integrate{\Omega}{}{\exp i \bm{\omega}^\top\left(\bm{x} + \pi\bm{n}\HadamardDiv\bm{W}\right) }{}{\bm{\omega}} &= \prod_{l=1}^d \integrate{-W_l}{W_l}{\exp i\left(x_l + \pi\frac{n_l}{W_l}\right)\omega_l}{}{\omega_l} \\
            &= \prod_{l=1}^d \frac{1}{i\left(x_l + \pi\frac{n_l}{W_l}\right)}\left[\exp i\left(x_l + \pi\frac{n_l}{W_l}\right)W_l - \exp (-i)\left(x_l + \pi\frac{n_l}{W_l}\right)W_l\right] \\
            &= \prod_{l=1}^d 2W_l \frac{\sin \left(x_l + \pi\frac{n_l}{W_l}\right)W_l}{\left(x_l + \pi\frac{n_l}{W_l}\right)W_l} = \prod_{l=1}^d 2W_l \prod_{l=1}^d \sinc W_l\left(x_l + \pi\frac{n_l}{W_l}\right)
        \end{align*}
        であるから
        \begin{align*}
            f(\bm{x}) &= \sum_{\bm{n}\in\integers^d} f\left(-\pi\bm{n}\HadamardDiv\bm{W}\right) \prod_{l=1}^d 2W_l \sinc \left(x_l + \pi\frac{n_l}{W_l}\right)W_l = \sum_{\bm{n}\in\integers^d} f\left(\pi\bm{n}\HadamardDiv\bm{W}\right) \prod_{l=1}^d \sinc W_l\left(x_l - \pi\frac{n_l}{W_l}\right) \\
            &= \sum_{\bm{n}\in\integers^d} f\left(\pi\frac{n_1}{W_1},\dots,\pi\frac{n_d}{W_d}\right) \prod_{l=1}^d \sinc W_l\left(x_l - \pi\frac{n_l}{W_l}\right)
        \end{align*}
    \end{proof}