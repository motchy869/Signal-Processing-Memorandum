\chapter{アップ・サンプリング}
    \section{アップ・サンプリングされた信号のDTFT}
        \newcommand{\xd}{x_\text{d}}
        \newcommand{\yd}{y_\text{d}}
        \newcommand{\Xd}{X_\text{d}}
        \newcommand{\Yd}{Y_\text{d}}
        \subsection{主張}
            記号を次のように定義する。
            \begin{itemize}
                \item $R\in\naturalNumbers,\;R\geq 2$ : アップ・サンプリング・レート
                \item $\xd:\integers\to\complexNumbers$ : 離散時間信号
                \item $\Ts>0$ : $\xd$ のサンプル周期
                \item $\yd$ : $\xd$ を $R$ 倍にアップ・サンプリングした離散時間信号。つまり $\yd(n) = \xd(n/R)\text{ for }R\mid n,\;0\text{ for }R\nmid n$ 。
                \item $\Xd$ : $\xd$ のDTFT
                \item $\Yd$ : $\yd$ のDTFT
            \end{itemize}
            このとき $\Yd(\omega) = \Xd(\omega)$ となる(アップ・サンプリング前の DTFT と完全に一致する)。
        \subsection{導出}
            \begin{align*}
                \Yd(\omega) &= \sum_{n=-\infty}^\infty \yd(n)\exp(-i\omega n\Ts/R) = \sum_{m=-\infty}^\infty \yd(mR)\exp(-i\omega (mR)\Ts/R) \\
                &= \sum_{m=-\infty}^\infty \xd(m)\exp(-i\omega m\Ts) = \Xd(\omega)
            \end{align*}
        \subsection{正規化角周波数で比較する}
            前述の通り $\Xd$ と $\Yd$ は完全に一致する。
            しかしフィルタ設計に於いてはしばしば第1 Nyquist 領域に関心がある、すなわち横軸が正規化角周波数で表されたスペクトラムに関心がある。
            この場合は $\Xd$ と $\Yd$ のグラフの見た目が異なる。\\
            両者をそれぞれ正規化角周波数で表示すると次式となる。
            \[ \tilde{\Xd}(\Omega) \coloneq \Xd(\Omega/\Ts),\quad \tilde{\Yd}(\Omega) \coloneq \Yd(\Omega/(\Ts/R)) = \Yd(R\Omega/\Ts) \]
            正規化角周波数で表示されたグラフでは、アップ・サンプリングされた信号のスペクトラムは元の信号のスペクトラムを横軸方向に $1/R$ に縮小した形になる。
        \subsection{数値例}
            サンプリング前の連続時間信号が $x(t) = e^{-\frac{n^2}{800}}\cos\left(\frac{\pi n}{5}\right)$ の場合の数値例が Mathematica ノートブック \verb|zero_pad_effect_for_DTFT.nb| にある。

